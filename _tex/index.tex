% Options for packages loaded elsewhere
\PassOptionsToPackage{unicode}{hyperref}
\PassOptionsToPackage{hyphens}{url}
\PassOptionsToPackage{dvipsnames,svgnames,x11names}{xcolor}
%
\documentclass[
  letterpaper,
  DIV=11,
  numbers=noendperiod]{scrartcl}

\usepackage{amsmath,amssymb}
\usepackage{iftex}
\ifPDFTeX
  \usepackage[T1]{fontenc}
  \usepackage[utf8]{inputenc}
  \usepackage{textcomp} % provide euro and other symbols
\else % if luatex or xetex
  \usepackage{unicode-math}
  \defaultfontfeatures{Scale=MatchLowercase}
  \defaultfontfeatures[\rmfamily]{Ligatures=TeX,Scale=1}
\fi
\usepackage{lmodern}
\ifPDFTeX\else  
    % xetex/luatex font selection
\fi
% Use upquote if available, for straight quotes in verbatim environments
\IfFileExists{upquote.sty}{\usepackage{upquote}}{}
\IfFileExists{microtype.sty}{% use microtype if available
  \usepackage[]{microtype}
  \UseMicrotypeSet[protrusion]{basicmath} % disable protrusion for tt fonts
}{}
\makeatletter
\@ifundefined{KOMAClassName}{% if non-KOMA class
  \IfFileExists{parskip.sty}{%
    \usepackage{parskip}
  }{% else
    \setlength{\parindent}{0pt}
    \setlength{\parskip}{6pt plus 2pt minus 1pt}}
}{% if KOMA class
  \KOMAoptions{parskip=half}}
\makeatother
\usepackage{xcolor}
\setlength{\emergencystretch}{3em} % prevent overfull lines
\setcounter{secnumdepth}{-\maxdimen} % remove section numbering
% Make \paragraph and \subparagraph free-standing
\ifx\paragraph\undefined\else
  \let\oldparagraph\paragraph
  \renewcommand{\paragraph}[1]{\oldparagraph{#1}\mbox{}}
\fi
\ifx\subparagraph\undefined\else
  \let\oldsubparagraph\subparagraph
  \renewcommand{\subparagraph}[1]{\oldsubparagraph{#1}\mbox{}}
\fi


\providecommand{\tightlist}{%
  \setlength{\itemsep}{0pt}\setlength{\parskip}{0pt}}\usepackage{longtable,booktabs,array}
\usepackage{calc} % for calculating minipage widths
% Correct order of tables after \paragraph or \subparagraph
\usepackage{etoolbox}
\makeatletter
\patchcmd\longtable{\par}{\if@noskipsec\mbox{}\fi\par}{}{}
\makeatother
% Allow footnotes in longtable head/foot
\IfFileExists{footnotehyper.sty}{\usepackage{footnotehyper}}{\usepackage{footnote}}
\makesavenoteenv{longtable}
\usepackage{graphicx}
\makeatletter
\def\maxwidth{\ifdim\Gin@nat@width>\linewidth\linewidth\else\Gin@nat@width\fi}
\def\maxheight{\ifdim\Gin@nat@height>\textheight\textheight\else\Gin@nat@height\fi}
\makeatother
% Scale images if necessary, so that they will not overflow the page
% margins by default, and it is still possible to overwrite the defaults
% using explicit options in \includegraphics[width, height, ...]{}
\setkeys{Gin}{width=\maxwidth,height=\maxheight,keepaspectratio}
% Set default figure placement to htbp
\makeatletter
\def\fps@figure{htbp}
\makeatother
% definitions for citeproc citations
\NewDocumentCommand\citeproctext{}{}
\NewDocumentCommand\citeproc{mm}{%
  \begingroup\def\citeproctext{#2}\cite{#1}\endgroup}
\makeatletter
 % allow citations to break across lines
 \let\@cite@ofmt\@firstofone
 % avoid brackets around text for \cite:
 \def\@biblabel#1{}
 \def\@cite#1#2{{#1\if@tempswa , #2\fi}}
\makeatother
\newlength{\cslhangindent}
\setlength{\cslhangindent}{1.5em}
\newlength{\csllabelwidth}
\setlength{\csllabelwidth}{3em}
\newenvironment{CSLReferences}[2] % #1 hanging-indent, #2 entry-spacing
 {\begin{list}{}{%
  \setlength{\itemindent}{0pt}
  \setlength{\leftmargin}{0pt}
  \setlength{\parsep}{0pt}
  % turn on hanging indent if param 1 is 1
  \ifodd #1
   \setlength{\leftmargin}{\cslhangindent}
   \setlength{\itemindent}{-1\cslhangindent}
  \fi
  % set entry spacing
  \setlength{\itemsep}{#2\baselineskip}}}
 {\end{list}}
\usepackage{calc}
\newcommand{\CSLBlock}[1]{\hfill\break\parbox[t]{\linewidth}{\strut\ignorespaces#1\strut}}
\newcommand{\CSLLeftMargin}[1]{\parbox[t]{\csllabelwidth}{\strut#1\strut}}
\newcommand{\CSLRightInline}[1]{\parbox[t]{\linewidth - \csllabelwidth}{\strut#1\strut}}
\newcommand{\CSLIndent}[1]{\hspace{\cslhangindent}#1}

\KOMAoption{captions}{tableheading}
\makeatletter
\@ifpackageloaded{tcolorbox}{}{\usepackage[skins,breakable]{tcolorbox}}
\@ifpackageloaded{fontawesome5}{}{\usepackage{fontawesome5}}
\definecolor{quarto-callout-color}{HTML}{909090}
\definecolor{quarto-callout-note-color}{HTML}{0758E5}
\definecolor{quarto-callout-important-color}{HTML}{CC1914}
\definecolor{quarto-callout-warning-color}{HTML}{EB9113}
\definecolor{quarto-callout-tip-color}{HTML}{00A047}
\definecolor{quarto-callout-caution-color}{HTML}{FC5300}
\definecolor{quarto-callout-color-frame}{HTML}{acacac}
\definecolor{quarto-callout-note-color-frame}{HTML}{4582ec}
\definecolor{quarto-callout-important-color-frame}{HTML}{d9534f}
\definecolor{quarto-callout-warning-color-frame}{HTML}{f0ad4e}
\definecolor{quarto-callout-tip-color-frame}{HTML}{02b875}
\definecolor{quarto-callout-caution-color-frame}{HTML}{fd7e14}
\makeatother
\makeatletter
\@ifpackageloaded{caption}{}{\usepackage{caption}}
\AtBeginDocument{%
\ifdefined\contentsname
  \renewcommand*\contentsname{Table of contents}
\else
  \newcommand\contentsname{Table of contents}
\fi
\ifdefined\listfigurename
  \renewcommand*\listfigurename{List of Figures}
\else
  \newcommand\listfigurename{List of Figures}
\fi
\ifdefined\listtablename
  \renewcommand*\listtablename{List of Tables}
\else
  \newcommand\listtablename{List of Tables}
\fi
\ifdefined\figurename
  \renewcommand*\figurename{Figure}
\else
  \newcommand\figurename{Figure}
\fi
\ifdefined\tablename
  \renewcommand*\tablename{Table}
\else
  \newcommand\tablename{Table}
\fi
}
\@ifpackageloaded{float}{}{\usepackage{float}}
\floatstyle{ruled}
\@ifundefined{c@chapter}{\newfloat{codelisting}{h}{lop}}{\newfloat{codelisting}{h}{lop}[chapter]}
\floatname{codelisting}{Listing}
\newcommand*\listoflistings{\listof{codelisting}{List of Listings}}
\makeatother
\makeatletter
\makeatother
\makeatletter
\@ifpackageloaded{caption}{}{\usepackage{caption}}
\@ifpackageloaded{subcaption}{}{\usepackage{subcaption}}
\makeatother
\ifLuaTeX
  \usepackage{selnolig}  % disable illegal ligatures
\fi
\usepackage{bookmark}

\IfFileExists{xurl.sty}{\usepackage{xurl}}{} % add URL line breaks if available
\urlstyle{same} % disable monospaced font for URLs
\hypersetup{
  pdftitle={Development and Validation of a Risk Prediction Model of linezolid-induced thrombocytopenia},
  colorlinks=true,
  linkcolor={blue},
  filecolor={Maroon},
  citecolor={Blue},
  urlcolor={Blue},
  pdfcreator={LaTeX via pandoc}}

\title{Development and Validation of a Risk Prediction Model of
linezolid-induced thrombocytopenia}
\author{Nhi Nguyen Ha \and An Tang Quoc \and Ha Tran Ngan \and Hang
Nguyen Thi Thu \and Hoa Vu Dinh \and Nhung TH Trinh \and Anh Nguyen
Hoang}
\date{Saturday, May 18, 2024}

\begin{document}
\maketitle
\begin{abstract}
Write abstract here, note the indentation
\end{abstract}

\section{Checklist}\label{checklist}

\begin{longtable}[]{@{}
  >{\raggedright\arraybackslash}p{(\columnwidth - 6\tabcolsep) * \real{0.0928}}
  >{\centering\arraybackslash}p{(\columnwidth - 6\tabcolsep) * \real{0.0176}}
  >{\raggedright\arraybackslash}p{(\columnwidth - 6\tabcolsep) * \real{0.8528}}
  >{\centering\arraybackslash}p{(\columnwidth - 6\tabcolsep) * \real{0.0336}}@{}}
\caption{TRIPOD+AI guidance for reporting clinical prediction models
that use regression or machine learning methods}\tabularnewline
\toprule\noalign{}
\begin{minipage}[b]{\linewidth}\raggedright
\textbf{Section/topic}
\end{minipage} & \begin{minipage}[b]{\linewidth}\centering
\textbf{Item}
\end{minipage} & \begin{minipage}[b]{\linewidth}\raggedright
\textbf{Description}
\end{minipage} & \begin{minipage}[b]{\linewidth}\centering
\textbf{Draft date}
\end{minipage} \\
\midrule\noalign{}
\endfirsthead
\toprule\noalign{}
\begin{minipage}[b]{\linewidth}\raggedright
\textbf{Section/topic}
\end{minipage} & \begin{minipage}[b]{\linewidth}\centering
\textbf{Item}
\end{minipage} & \begin{minipage}[b]{\linewidth}\raggedright
\textbf{Description}
\end{minipage} & \begin{minipage}[b]{\linewidth}\centering
\textbf{Draft date}
\end{minipage} \\
\midrule\noalign{}
\endhead
\bottomrule\noalign{}
\endlastfoot
\textbf{TITLE} & & & \\
\emph{Title} & 1 & \begin{minipage}[t]{\linewidth}\raggedright
\textbf{Identify the study as developing or evaluating the performance
of a multivariable prediction model, the target population, and the
outcome to be predicted}

\begin{itemize}
\item
  \emph{Informative titles aid the identification of prediction model
  studies by potential readers and also systematic reviewers}
\item
  \emph{Report an informative title that provides key information about
  the target population and the outcome being predicted by the model}
\end{itemize}
\end{minipage} & \\
\textbf{ABSTRACT} & & & \\
\emph{Abstract} & 2 & \begin{minipage}[t]{\linewidth}\raggedright
\textbf{See TRIPOD+AI for Abstracts checklist}

\begin{itemize}
\tightlist
\item
  \emph{Report an abstract addressing each item in the TRIPOD+AI for
  Abstracts checklist}
\end{itemize}
\end{minipage} & \\
\textbf{INTRODUCTION} & & & \\
\emph{Background} & 3a & \begin{minipage}[t]{\linewidth}\raggedright
\textbf{Explain the healthcare context (including whether diagnostic or
prognostic) and rationale for developing or evaluating the prediction
model, including references to existing models}

\begin{itemize}
\item
  \emph{Describe the healthcare setting where the model is intended to
  be used or needed}
\item
  \emph{Where an existing prediction model is available, provide a clear
  justification for developing a new model}
\item
  \emph{For studies evaluating an existing model, provide the rationale
  for the evaluation, and provide references to all models being
  evaluated}
\end{itemize}
\end{minipage} & \\
& 3b & \begin{minipage}[t]{\linewidth}\raggedright
\textbf{Describe the target population and the intended purpose of the
prediction model in the context of the care pathway, including its
intended users (e.g., healthcare professionals, patients, public)}

\begin{itemize}
\item
  \emph{Describe who is the target population for the developed or
  evaluated model, e.g., people of a certain age, in a specific country,
  or with a specific disease}
\item
  \emph{Describe the intended purpose of the model, including the
  clinical decision or guidance the model is intended used to support
  (e.g., referral for further testing or hospital admission, triage,
  starting a treatment, or changing a lifestyle) and the point in the
  care pathway the model is to be intended used}
\item
  \emph{Describe who the intended users of the model are, and if the
  model is for healthcare professionals, patients, public or other}
\end{itemize}
\end{minipage} & \\
& 3c & \begin{minipage}[t]{\linewidth}\raggedright
\textbf{Describe any known health inequalities between sociodemographic
groups}

\begin{itemize}
\tightlist
\item
  \emph{In the context of the healthcare setting where the model is
  intended to be used, describe any known health inequalities between
  sociodemographic groups in the target population (along with citations
  to support the health inequalities)}
\end{itemize}
\end{minipage} & \\
\emph{Objectives} & 4 & \begin{minipage}[t]{\linewidth}\raggedright
\textbf{Specify the study objectives, including whether the study
describes the development or validation of a prediction model (or both)}

\begin{itemize}
\tightlist
\item
  \emph{Provide an explicit statement of all objectives of the study,
  describing whether the study is developing a prediction model,
  evaluating the performance of a prediction model, or both}
\end{itemize}
\end{minipage} & \\
\textbf{METHODS} & & & \\
\emph{Data} & 5a & \begin{minipage}[t]{\linewidth}\raggedright
\textbf{Describe the sources of data separately for the development and
evaluation datasets (e.g., randomised trial, cohort, routine care or
registry data), the rationale for using these data, and
representativeness of the data}

\begin{itemize}
\item
  \emph{Provide a description of the source of the data used for model
  development and evaluation of model performance, including whether the
  data are (for example) from a randomised trial, a cohort, a registry
  or from electronic routine healthcare records}
\item
  \emph{Specify whether the study is using existing data or is
  prospectively collecting new data for the purpose of the prediction
  model study}
\item
  \emph{Where existing data are being used (i.e., they were originally
  collected for a different purpose), provide the rationale for using
  these data, and comment on the suitability (particularly if data are
  being used from a different setting or country to the intended target
  population) and representativeness of these data with respect to the
  intended target population and context}
\item
  \emph{A description of the data sources should be provided for all
  data sets, and separately for development and evaluation}
\item
  \emph{If any synthetic data have been used, then provide reasons as to
  why, and provide all details on how the synthetic data have been
  created (and code, see item 18f) and used in the study}
\end{itemize}
\end{minipage} & \\
& 5b & \begin{minipage}[t]{\linewidth}\raggedright
\textbf{Specify the dates of the collected participant data, including
start and end of participant accrual; and, if applicable, end of
follow-up}

\begin{itemize}
\item
  \emph{Specify the start and end dates of the period for which the
  participants or the used data were selected}
\item
  \emph{For models predicting prognosis, the duration of follow-up is
  important so report the date of end of follow-up}
\end{itemize}
\end{minipage} & \\
\emph{Participants} & 6a & \begin{minipage}[t]{\linewidth}\raggedright
\textbf{Specify key elements of the study setting (e.g., primary care,
secondary care, general population) including the number and location of
centres}

\begin{itemize}
\item
  \emph{Describe the healthcare setting, and where the participants in
  the study were recruited from}
\item
  \emph{Report the geographical location (at a minimum, the country) and
  centres (including the number of centres) of the study}
\end{itemize}
\end{minipage} & \\
& 6b & \begin{minipage}[t]{\linewidth}\raggedright
\textbf{Describe the eligibility criteria for study participants}

\begin{itemize}
\item
  \emph{The eligibility criteria for participants should be reported to
  understand the potential applicability and generalisability of the
  prediction model}
\item
  \emph{This includes reporting any restrictions of continuous
  variables, e.g., age range used to define the eligibility of the
  included participants}
\end{itemize}
\end{minipage} & \\
& 6c & \begin{minipage}[t]{\linewidth}\raggedright
\textbf{Give details of any treatments received, and how they were
handled during model development or evaluation, if relevant}

\begin{itemize}
\item
  \emph{Any treatments received before or at the start of follow-up
  should be reported, and whether and how this was handled during the
  development or evaluation of the prediction model (if relevant)}
\item
  \emph{Any treatments received between the moment the prediction model
  is used and the measurement of the outcome, that could modify the
  probability of the outcome, should be reported (if relevant)}
\end{itemize}
\end{minipage} & \\
\emph{Data preparation} & 7 &
\begin{minipage}[t]{\linewidth}\raggedright
\textbf{Describe any data pre-processing and quality checking, including
whether this was similar across relevant sociodemographic groups}

\begin{itemize}
\item
  \emph{Describe any data cleaning steps, this includes any feature
  engineering, transformation of raw data, feature reduction and data
  quality checks. All code used for data cleaning should be made
  available (see item 18f)}
\item
  \emph{For analyses using data from multiple sources (e.g., data from
  different studies, cohorts, or registries), describe any harmonisation
  (e.g., of outcome and predictors)}
\item
  \emph{Confirm whether all data pre-processing/data cleaning steps were
  similar across key sociodemographic groups, if relevant}
\item
  \emph{If the data pre-processing/data cleaning steps are extensive,
  consider reporting this information in the supplementary material}
\end{itemize}
\end{minipage} & \\
& C-7b & \textbf{Describe the method for assessing risk of bias and
applicability in the individual clusters (eg, using PROBAST)} & \\
\emph{Outcomes} & 8a & \begin{minipage}[t]{\linewidth}\raggedright
\textbf{Clearly define the outcome that is being predicted and the time
horizon, including how and when assessed, the rationale for choosing
this outcome, and whether the method of outcome assessment is consistent
across sociodemographic groups}

\begin{itemize}
\item
  \emph{For diagnostic prediction models, the outcome should be clearly
  defined, including whether a (widely accepted) reference standard
  (ground truth) was used to determine the presence or absence of the
  outcome}
\item
  \emph{For prognostic models, i.e., models predicting an outcome in the
  future, authors should report the time-horizon of the outcome
  prediction. For example, predicting the 28-day risk of mortality
  following cardiothoracic surgery, or the 10-year risk of fractures in
  patients with osteoporosis. Also, the frequency of outcome assessment
  during follow-up should be reported}
\item
  \emph{If standard definitions are used, e.g., using ICD codes, this
  should be stated and referenced}
\item
  \emph{Any discrepancies in the outcome assessment across
  socio-demographic groups should be reported}
\item
  \emph{In some instances, it may be necessary to confirm that no
  predictors were used to define the outcome or are a proxy for the
  outcome}
\end{itemize}
\end{minipage} & \\
& 8b & \begin{minipage}[t]{\linewidth}\raggedright
\textbf{If outcome assessment requires subjective interpretation,
describe the qualifications and demographic characteristics of the
outcome assessors}

\begin{itemize}
\item
  \emph{For outcomes that require a subjective interpretation (e.g.,
  interpreting the results from an imaging test, describe the number,
  qualification, and demographic characteristics of the outcome
  assessors)}
\item
  \emph{If the measurement and interpretation of the outcome require
  (additional) training or specific instructions, these should be
  reported.}
\item
  \emph{If extensive, consider reporting this information in the
  supplementary material}
\end{itemize}
\end{minipage} & \\
& 8c & \begin{minipage}[t]{\linewidth}\raggedright
\textbf{Report any actions to blind assessment of the outcome to be
predicted}

\begin{itemize}
\item
  \emph{The outcome being predicted should be assessed blind to
  information about the predictors -- particularly relevant for outcomes
  requiring a subjective interpretation thereby avoiding data (label)
  leakage}
\item
  \emph{If appropriate, authors should describe which information was
  available to the outcome assessors and report any specific actions to
  blinding the outcome assessment}
\end{itemize}
\end{minipage} & \\
\emph{Predictors} & 9a & \begin{minipage}[t]{\linewidth}\raggedright
Describe the choice of initial predictors (e.g., literature, previous
models, all available predictors) and any pre-selection of predictors
before model building

\begin{itemize}
\item
  Provide details on how the initial list of predictors were considered
  for inclusion in the model building, and whether they were chosen
  based on a (systematic) review of the literature, clinical input
  (domain experts), or simply whether using all predictors in the
  available data
\item
  If any pre-selection of predictors, before model building, was carried
  out, then provide details how this was done. For example, were
  predictors omitted for model building due to high amounts of missing
  data, or predictors not considered plausibly (clinically) related to
  the outcome being predicted
\item
  The list of initial predictors may be extensive, in these instances
  reporting these in the supplementary material is advisable
\end{itemize}
\end{minipage} & \\
& 9b & \begin{minipage}[t]{\linewidth}\raggedright
\textbf{Clearly define all predictors, including how and when they were
measured (and any actions to blind assessment of predictors for the
outcome and other predictors)}

\begin{itemize}
\item
  \emph{All predictors included in the modelling should be clearly
  defined, along with units of measurement, and all categories for
  categorical predictors, so that readers and others can replicate,
  implement, or evaluate the performance of the model}
\item
  \emph{Details on how and when the predictor values were measured. Note
  that predictors should be measured before or at the time the model is
  intended to be used}
\item
  \emph{For predictors requiring subjective interpretation, it may be
  important to interpret this blind to the values of other predictors
  considered in the modelling (e.g., avoiding data leakage). Authors
  should report any actions to blind the assessment of the predictor
  measurement to other predictors}
\item
  \emph{Specifically for diagnostic models, the measurement of the
  predictors should be done without knowledge of the outcome of the
  individual as this could artificially inflate the association between
  the predictors and the outcome. Authors should report any actions to
  blind the assessment of the predictor measurements to the outcome
  value}
\item
  \emph{In some instances, the number of predictors can be very large
  and thus reporting them all in the main manuscript is unhelpful, in
  these instances, it is still important to clearly define all the
  predictors, and reporting this in the supplementary material should be
  considered}
\end{itemize}
\end{minipage} & \\
& 9c & \begin{minipage}[t]{\linewidth}\raggedright
\textbf{If predictor measurement requires subjective interpretation,
describe the qualifications and demographic characteristics of the
predictor assessors}

\begin{itemize}
\item
  \emph{For predictors that require a subjective interpretation (e.g.,
  interpreting the results from an imaging test), the qualifications and
  demographic characteristics of the predictor assessors should be
  reported}
\item
  \emph{If the measurement and interpretation require (additional)
  training or specific instructions, then these should be reported. This
  could be reported in the supplementary material}
\end{itemize}
\end{minipage} & \\
\emph{Sample size} & 10 & \begin{minipage}[t]{\linewidth}\raggedright
\textbf{Explain how the study size was arrived at (separately for
development and evaluation), and justify that the study size was
sufficient to answer the research question. Include details of any
sample size calculation}

\begin{itemize}
\item
  \emph{Describe how the sample size was determined -- this should be
  done separately for determining the sample size needed for model
  development and the sample size needed to evaluate the performance of
  the model irrespective of whether data are being prospectively
  collected or using existing data}
\item
  \emph{Provide details and all estimates used in any sample size
  calculation}
\item
  \emph{If no formal sample size calculation was done, e.g., all
  available data were used, provide a justification whether the size of
  the data was sufficient to answer the research question}
\end{itemize}
\end{minipage} & \\
\emph{Missing data} & 11 & \begin{minipage}[t]{\linewidth}\raggedright
\textbf{Describe how missing data were handled. Provide reasons for
omitting any data}

\begin{itemize}
\item
  \emph{Missing data is an omnipresent problem. Authors should report
  for each predictor being considered for inclusion in the model the
  number of missing values}
\item
  \emph{The handling of missing values should be reported, including any
  assumptions for the reason of the missingness}
\item
  \emph{If individuals (or predictors) have been omitted due to the
  missing values, this should be reported, and reasons given}
\item
  \emph{If missing values have been imputed, then full details of the
  method for imputing any missing values should be reported}
\item
  \emph{If missing values have been imputed confirm it was done
  separately for the training and any test data (i.e., avoiding
  leakage)}
\end{itemize}
\end{minipage} & \\
\emph{Analytical methods} & 12a &
\begin{minipage}[t]{\linewidth}\raggedright
\textbf{Describe how the data were used (e.g., for development and
evaluation of model performance) in the analysis, including whether the
data were partitioned, considering any sample size requirements}

\begin{itemize}
\item
  \emph{Describe how the available data were used to develop the model
  and to evaluate model performance, including whether and how the data
  were partitioned, and the reasons for partitioning the data (e.g.,
  model development, hyperparameter tuning, evaluating model
  performance, internal-external cross-validation)}
\item
  \emph{If the data has been partitioned, report whether sample size
  requirements (see item 10) were considered during the partitioning,
  and whether the size of the partitioned data are sufficient to carry
  out the analyses and answer the research question}
\item
  \emph{If the data has been partitioned into training (including any
  hyperparameter tuning data) and test data, confirm that there has been
  no data leakage}
\item
  \emph{If the data contain multiple records or samples from the same
  individual, and if the data has been partitioned into training
  (including any hyperparameter tuning data) and test data, confirm
  there has been no leakage of individuals across any of the partitioned
  data or if not, how describe how this was handled in the analysis (see
  item 12c)}
\end{itemize}
\end{minipage} & \\
& 12b & \begin{minipage}[t]{\linewidth}\raggedright
\textbf{Depending on the type of model, describe how predictors were
handled in the analyses (functional form, rescaling, transformation, or
any standardisation)}

\begin{itemize}
\item
  \emph{For any predictors that have been transformed during the
  analysis, i.e., rescaled, or standardised, describe how this was done}
\item
  \emph{For any categorical predictors, where collapsing of categories
  has been carried out, e.g., due to small sample size/too few outcome
  events, provide the details and reasons}
\end{itemize}
\end{minipage} & \\
& 12c & \begin{minipage}[t]{\linewidth}\raggedright
\textbf{Specify the type of model, rationale (separately for all model
buiding approaches), all model-building steps, including any
hyperparameter tuning, and method for internal validation}

\begin{itemize}
\item
  \emph{Clearly specify the type of model (or models) being developed
  (e.g., logistic regression, Cox regression, random forest, neural
  network) and provide a rationale for using each model building method
  -- consider the type of outcome being predicted and how the prediction
  model will be implemented in practice}
\item
  \emph{For each model, clearly describe all the steps in the model
  building, including any hyperparameter tuning, what hyperparameters
  have been tuned and how this was done. If many model building
  approaches are being applied and word limits prohibit a full
  description, then use supplementary material to provide the details}
\item
  \emph{For studies that are developing more than one model (e.g., using
  different model building methods), clearly describe the criteria to
  choose which is the model being put forward (if any), see item 12e and
  item 23 on model performance)}
\item
  \emph{The internal validation approach (to evaluate model performance)
  during model development should be clearly described, e.g., was k-fold
  cross validation or bootstrapping used. Clarify whether all model
  building steps (including hyperparameter tuning) was replayed during
  the method of internal evaluation}
\item
  \emph{Clearly describe any methods (e.g., bootstrapping) used to
  examine model stability (e.g., in terms of predictor selection,
  predictive performance and individual predictions) (Riley \& Collins,
  Biom J 2023; 65: 2200302 {[}DOI: 10.1002/bimj.202200302{]})}
\item
  \emph{If the data contain multiple records or samples from the same
  individual, describe how this was handled in the model building and
  internal validation (e.g., if k-fold cross-validation was used,
  confirm if all records/samples for an individual were included in the
  same fold (e.g., avoiding data leakage)}
\end{itemize}
\end{minipage} & \\
& 12d & \begin{minipage}[t]{\linewidth}\raggedright
\textbf{Describe if and how any heterogeneity in estimates of model
parameter values and model performance was handled and quantified across
clusters (e.g., hospitals, countries). See TRIPOD-Cluster for additional
considerations}

\begin{itemize}
\item
  \emph{If the analysis has accounted for any clustering in the data
  (e.g., from combining individual participant data from multiple
  studies, or data clustered by medical centre/hospital, or country)
  during the model development or evaluation of model performance, the
  rationale and methods used to account for clustering should be clearly
  described}
\item
  \emph{For specific reporting recommendations for prediction model
  studies that have accounted for clustering and heterogeneity in model
  parameter values and performance, authors should consult the
  TRIPOD-Cluster checklist (Debray et al, BMJ 2023; 380: e071018 {[}DOI:
  10.1136/bmj-2022-071018{]})}
\end{itemize}
\end{minipage} & \\
& 12e & \begin{minipage}[t]{\linewidth}\raggedright
\textbf{Specify all measures and plots used (and their rationale) to
evaluate model performance (e.g., discrimination, calibration, clinical
utility) and, if relevant, to compare multiple models}

\begin{itemize}
\item
  \emph{Report all the measures used to evaluate model performance. It
  is generally expected that as a minimum, model discrimination and
  calibration (including calibration plots) are presented}
\item
  \emph{If the prediction model is predicting a time-to-event outcome,
  then clearly describe the measures and methods that have been used to
  account for the time-to-event nature (i.e., censoring). Similarly, the
  handling of any competing risks should also be stated (if applicable)}
\item
  \emph{For prognostic models, report all time-points at which the
  model's predictive performance was evaluated}
\item
  \emph{Report the methods used for graphical displays of model
  performance, such as calibration plots (with smooth calibration
  curves) and decision curves}
\item
  \emph{If multiple models are being compared, i.e., comparing against
  an existing model or comparing multiple modelling approaches, then the
  methods used for comparing these models, and the criteria for making
  any judgements on superior performance should be clearly explained}
\end{itemize}
\end{minipage} & \\
& 12f & \begin{minipage}[t]{\linewidth}\raggedright
\textbf{Describe any model updating (e.g., recalibration) arising from
the model evaluation, either overall or for particular sociodemographic
groups or settings}

\begin{itemize}
\tightlist
\item
  \emph{If the model is updated following the validation, such as
  recalibration or refitting -- whether in the entire cohort or in a
  specific sociodemographic group, then provide details on the methods
  used to update the model}
\end{itemize}
\end{minipage} & Inapplicable \\
& 12g & \begin{minipage}[t]{\linewidth}\raggedright
\textbf{For model evaluation, describe how the model predictions were
calculated (e.g., formula, code, object, application programming
interface)}

\begin{itemize}
\item
  \emph{For studies evaluating an existing model in a separate data set
  (i.e., an external validation study), provide details on how the
  individual predictions from the model were calculated. If a model is
  not freely/publicly available, explain how the predictions were
  obtained}
\item
  \emph{If a regression model equation was being evaluated, provide
  details of this equation (e.g., consider presenting this equation,
  provide a citation to the original study that developed the equation)}
\item
  \emph{For studies evaluating a prediction model where there is no
  equation (e.g., a neural network, random forest), provide details on
  how the predictions were made, e.g., code, software object, API, and
  where can this be found (i.e., URL, DOI)}
\item
  \emph{If individual predictions from the model were used to create
  risk groups or classifications (that were not specified in the model
  development) then details on how and why this was done should be
  reported (see item 15)}
\end{itemize}
\end{minipage} & Inapplicable \\
\emph{Class imbalance} & 13 &
\begin{minipage}[t]{\linewidth}\raggedright
\textbf{If class imbalance methods were used, state why and how this was
done, and any subsequent methods to recalibrate the model or the model
predictions}

\begin{itemize}
\item
  \emph{If class imbalance methods (e.g., under/over sampling, SMOTE6)
  have been used, then provide a rationale for doing so, and how this
  was done -- considering any impact on sample size (e.g., for
  undersampling methods)}
\item
  \emph{Imbalance corrections have an impact on model calibration (van
  den Goorbergh et al, J Am Med Inform Assoc 2022; 29: 1525--1534
  {[}DOI: 10.1093/jamia/ocac093{]}), yielding probability estimates that
  are too high (which also has an impact on defining any risk groups),
  describe the methods used to recalibrate the model or the model
  predictions}
\end{itemize}
\end{minipage} & Inapplicable \\
\emph{Fairness} & 14 & \begin{minipage}[t]{\linewidth}\raggedright
\textbf{Describe any approaches that were used to address model fairness
and their rationale}

\begin{itemize}
\item
  \emph{Fairness refers to ensuring that a prediction model does not
  discriminate against individuals or groups, for example based on
  personal attributes such as race, gender, age and all approaches used
  to address fairness should be clearly explained along with their
  rationale}
\item
  \emph{It is important to ensure the data contains representative
  groups (of the target population) when developing the model and
  evaluating its performance and researchers should attempt to
  demonstrate this}
\item
  \emph{If the prediction model is developed using data with
  underrepresented groups or particular groups not included, then
  evaluation in these groups in representative data is needed to
  evaluate the model in these groups, as to increase generalisability to
  more groups of individuals beyond those in the development and
  evaluation data}
\end{itemize}
\end{minipage} & \\
\emph{Model output} & 15 & \begin{minipage}[t]{\linewidth}\raggedright
\textbf{Specify the output of the prediction model (e.g., probabilities,
classification). Provide details and rationale for any classification
and how the thresholds were identified}

\begin{itemize}
\item
  \emph{Most models output a probability estimate for an individual,
  whilst some models turn the output into a classification (e.g., into
  low or high-risk groups), this should be clearly stated. If
  classification or risk groups have been created, then the rationale
  for doing so in the context of the care pathway and how these risk
  groups inform any clinical decisions should be made}
\item
  \emph{For models producing a classification or risk groups, this
  should be clearly reported, and any thresholds (e.g., range of
  estimated probabilities defining the groups) should be specified
  (whether these are based on the literature, clinical guidelines,
  statistical considerations or ad-hoc)}
\item
  \emph{If uncertainty intervals for individual prediction model outputs
  have been presented then provide details on how this was done (e.g.,
  using the variance-covariance matrix of parameter estimates or
  conformal prediction)}
\end{itemize}
\end{minipage} & \\
\emph{Development versus evaluation} & 16 &
\begin{minipage}[t]{\linewidth}\raggedright
\textbf{Identify any differences between the development and evaluation
data in healthcare setting, eligibility criteria, outcome, and
predictors}

\begin{itemize}
\tightlist
\item
  \emph{Prediction models developed in one setting, centre or country
  are not necessarily useful in a different setting, centre, or country.
  Eligibility criteria, outcome and predictors definitions might
  (intentionally) differ between data from different sources. Describing
  any differences between the development data and data used to evaluate
  model performance is useful to understand and interpret the
  performance and generalisability of the model in the context of the
  original model development data}
\end{itemize}
\end{minipage} & Inapplicable (yet) \\
\emph{Ethical approval} & 17 &
\begin{minipage}[t]{\linewidth}\raggedright
\textbf{Name the institutional research board or ethics committee that
approved the study and describe the participant-informed consent or the
ethics committee waiver of informed consent}

\begin{itemize}
\tightlist
\item
  \emph{If the study has no institutional research board or ethics
  approval, then clearly state so, with reasons}
\end{itemize}
\end{minipage} & \\
\textbf{OPEN SCIENCE} & & & \\
\emph{Funding} & 18a & \begin{minipage}[t]{\linewidth}\raggedright
\textbf{Give the source of funding and the role of the funders for the
present study}

\begin{itemize}
\item
  \emph{Provide details on whether the study was funded and provide any
  details on the role the funder had in the study.}
\item
  \emph{Provide any additional funding sources of all authors}
\end{itemize}
\end{minipage} & \\
\emph{Conflicts of interest} & 18b &
\begin{minipage}[t]{\linewidth}\raggedright
\textbf{Declare any conflicts of interest and financial disclosures for
all authors}

\begin{itemize}
\tightlist
\item
  \emph{Disclose any of the authors' relationships or activities that
  readers could consider pertinent or that may have influenced the study
  design, conduct, interpretation, or reporting}
\end{itemize}
\end{minipage} & \\
\emph{Protocol} & 18c & \begin{minipage}[t]{\linewidth}\raggedright
\textbf{Indicate where the study protocol can be accessed or state that
a protocol was not prepared}

\begin{itemize}
\item
  \emph{Provide all details on the availability of the study protocol,
  including where the study protocol can be found (e.g., publication
  details, in supplementary material, publicly available in a repository
  such as on the Open Science Framework), including a URL or DOI}
\item
  \emph{Clearly state if no study protocol was developed or publicly
  available (and reasons)}
\item
  \emph{If there are any notable deviations from what was specified in
  the study protocol, provide a summary and reasons for the deviation}
\end{itemize}
\end{minipage} & \\
\emph{Registration} & 18d & \begin{minipage}[t]{\linewidth}\raggedright
\textbf{Provide registration information for the study, including
register name and registration number, or state that the study was not
registered}

\begin{itemize}
\item
  \emph{If the study has been registered (e.g., on clinicaltrials.gov,
  Open Science Framework), then provide details on the registration
  number, the name of the register and a link to the registration
  (including any DOI)}
\item
  \emph{Clearly state if the study has not been registered}
\end{itemize}
\end{minipage} & \\
\emph{Data sharing} & 18e & \begin{minipage}[t]{\linewidth}\raggedright
\textbf{Provide details of the availability of the study data}

\begin{itemize}
\item
  \emph{Provide details on the availability of the study data, including
  where the data can be found (e.g., public repository, URL, DOI), how
  it can be retrieved, any conditions or restrictions on obtaining and
  using the data. A data dictionary should accompany any shared data.}
\item
  \emph{If data cannot be shared, provide reasons as to why}
\item
  \emph{Avoid platitudes such as `Data available upon reasonable
  request' without specifying conditions for what constitutes a
  reasonable request}
\end{itemize}
\end{minipage} & \\
\emph{Code sharing} & 18f & \begin{minipage}[t]{\linewidth}\raggedright
\textbf{Provide details of the availability of the analytical code
(e.g., any data cleaning, feature engineering, model building,
evaluation)}

\begin{itemize}
\item
  \emph{Provide all details on the availability of the analytical code
  (and documentation on how to run the code), including where the code
  can be found (e.g., code repository, DOI, link), how it can be
  retrieved, any conditions or licences to obtain and use the code
  should be reported (and version)}
\item
  \emph{The analytical code is all code needed to replicate (in
  principle) all the reported results and findings of the study
  (including any code for data cleaning). The software and any packages
  needed to reproduce (in principle) the study findings should be
  reported (including any version numbers). In some instances, more
  details on the computing environment may need to be reported (e.g.,
  hardware, operating system, CPU, RAM)}
\end{itemize}
\end{minipage} & \\
\textbf{PATIENT \& PUBLIC INVOLVEMENT} & & & \\
\emph{Patient \& Public Involvement} & 19 &
\begin{minipage}[t]{\linewidth}\raggedright
\textbf{Provide details of any patient and public involvement during the
design, conduct, reporting, interpretation, or dissemination of the
study or state no involvement}

\begin{itemize}
\item
  \emph{Describe how patients or public were involved in the planning,
  design, conduct, reporting or dissemination of the study and its
  findings.}
\item
  \emph{Were the findings of the study presented to patients or the
  public?}
\item
  \emph{Consider using the GRIPP2 statement to report patient and public
  involvement in the research (Staniszewska et al, BMJ 2017; j3453
  {[}DOI: 10.1136/bmj.j3453{]})}
\item
  \emph{If no patients or public were involved in any aspect of the
  study, then clearly state so}
\end{itemize}
\end{minipage} & \\
\textbf{RESULTS} & & & \\
\emph{Participants} & 20a & \begin{minipage}[t]{\linewidth}\raggedright
\textbf{Describe the flow of participants through the study, including
the number of participants with and without the outcome and, if
applicable, a summary of the follow-up time. A diagram may be helpful}

\begin{itemize}
\item
  \emph{A flow diagram can be useful to describe the flow of
  participants through a study, where the entry point to the flow
  diagram is the source of participants, and then successive steps can
  relate to eligibility criteria, follow-up (if applicable). and data
  availability}
\item
  \emph{Other useful information to present in the flow diagram include
  the number of participants with missing values, and the number of
  outcome events}
\item
  \emph{For studies of prognosis or diagnosis with delayed reference
  testing, a summary of the follow-up time should be reported (e.g.,
  median follow-up, and range)}
\end{itemize}
\end{minipage} & \\
& 20b & \begin{minipage}[t]{\linewidth}\raggedright
\textbf{Report the characteristics overall and, where applicable, for
each data source or setting, including the key dates, key predictors
(including demographics), treatments received, sample size, number of
outcome events, follow-up time, and amount of missing data. A table may
be helpful. Report any differences across key demographic groups}

\begin{itemize}
\item
  \emph{Report, possibly using a table, a summary of all data sets used,
  including the distribution of outcomes, predictors (e.g., mean/median,
  standard deviation/interquartile range, frequency), any treatments
  received, the sample size (and number of outcome events, summary of
  the follow-up time, and for each predictor, the number and proportion
  of missing values}
\item
  \emph{If relevant, it may be useful to report any differences across
  key demographic groups of interest}
\end{itemize}
\end{minipage} & \\
& 20c & \begin{minipage}[t]{\linewidth}\raggedright
\textbf{For model evaluation, show a comparison with the development
data of the distribution of important predictors (demographics,
predictors, and outcome).}

\begin{itemize}
\tightlist
\item
  \emph{For studies evaluating the performance of an existing model
  (including those within a model development study) provide a
  comparison of the distribution of important variables (e.g.,
  mean/median, standard deviation/interquartile range, frequency), such
  as demographics, predictors in the model, and outcome, including
  proportion of missing values. This is probably best presented in a
  table and consider reporting this by outcome status}
\end{itemize}
\end{minipage} & Inapplicable \\
\emph{Risk of bias} & C-11 & \textbf{Report the results of the
risk-of-bias assessment in the individual clusters} & \\
\emph{Model development} & 21 &
\begin{minipage}[t]{\linewidth}\raggedright
\textbf{Specify the number of participants and outcome events in each
analysis (e.g., for model development, hyperparameter tuning, model
evaluation)}

\begin{itemize}
\item
  \emph{The sample size (including the number of outcome events) should
  be reported for each analysis (i.e., each model development, each
  model evaluation), as they can often vary across different analyses in
  a prediction model study (e.g., after data partitioning, model
  hyperparameter tuning), and particularly in the presence of missing
  data}
\item
  \emph{If the data contain multiple samples or records for an
  individual report also report the number of individuals}
\end{itemize}
\end{minipage} & \\
\emph{Model specification} & 22 &
\begin{minipage}[t]{\linewidth}\raggedright
\textbf{Provide details of the full prediction model (e.g., formula,
code, object, API) to allow predictions in new individuals and to enable
third-party evaluation and implementation, including any restrictions to
access or re-use (e.g., freely available, proprietary)}

\begin{itemize}
\item
  \emph{The `product' of a prediction model development study is the
  prediction model. It is therefore important to provide details on the
  model, and how it can be used to allow predictions for new individuals
  to be made. For example, provide the equation for a regression model,
  for models developed using methods where the model cannot be `written
  down' as an equation, provide details on the availability of code,
  software objects or API so that others can evaluate this model in
  their own data, or implement it in daily practice}
\item
  \emph{If multiple models have been developed, then provide details on
  the availability of all models}
\item
  \emph{Explain how to use the model to allow others to make predictions
  in new individuals.}
\item
  \emph{Provide details of any hardware requirements, and software (and
  packages) to enable third-party testing, implementation and
  monitoring}
\item
  \emph{If a model cannot be made publicly available (e.g., for
  commercial reasons), this should be clearly reported, and any
  conditions on gaining access to the model to enable predictions to be
  calculated for new individuals and third-party evaluation should be
  reported}
\end{itemize}
\end{minipage} & \\
\emph{Model performance} & 23a &
\begin{minipage}[t]{\linewidth}\raggedright
\textbf{Report model performance estimates with confidence intervals,
including for any key subgroups (e.g., sociodemographic). Consider plots
to aid presentation}

\begin{itemize}
\item
  \emph{Estimates of all model performance measures described in item
  12e should be presented along with confidence intervals.}
\item
  \emph{Report model performance estimates for the overall population
  and for any key groups (e.g., sex, ethnicity) of interest (e.g., as
  part of fairness checks) with confidence intervals}
\item
  \emph{Use plots to present and aid evaluation, such as calibration
  plots (with smooth calibration curves and distributions of predicted
  values) and decision curves}
\item
  \emph{Report performance estimates for all evaluations undertaken
  (e.g., in development data; in evaluation data; from internal
  validation process, etc), including at each time-point examined (for
  prognostic models)}
\item
  \emph{Report any examinations of model stability, e.g., in terms of
  performance estimates and variability of individual predictions across
  models developed in bootstrap samples (Riley \& Collins, Biom J 2023;
  65: 2200302 {[}DOI: 10.1002/bimj.202200302{]})}
\item
  \emph{Clearly indicate which data have been used to present each
  performance estimate}
\end{itemize}
\end{minipage} & \\
& 23b & \begin{minipage}[t]{\linewidth}\raggedright
\textbf{If examined, report results of any heterogeneity in model
performance across clusters. See TRIPOD Cluster for additional details}

\begin{itemize}
\item
  \emph{If the evaluation of model performance has accounted for any
  clustering in the data (e.g., from combining individual participant
  data from multiple studies, or data clustered by centre/hospital, or
  country), the results should be reported, along with confidence
  intervals (see item 23a)}
\item
  \emph{For specific reporting recommendations for prediction model
  studies that have accounted for clustering and heterogeneity in model
  performance, authors should consult the TRIPOD-Cluster checklist
  (Debray et al, BMJ 2023; 380: e071018 {[}DOI: 10.1136/bmj-
  2022-071018{]})}
\end{itemize}
\end{minipage} & \\
\emph{Model updating} & 24 & \begin{minipage}[t]{\linewidth}\raggedright
\textbf{Report the results from any model updating, including the
updated model and subsequent performance}

\begin{itemize}
\item
  \emph{If the prediction model has been updated (e.g., recalibrated,
  re-fit) following the validation, details of the updated prediction
  model to enable third-party evaluation and implementation, including
  any restrictions to access or re-use should be reported (see item 22)}
\item
  \emph{The performance of the updated model should be reported (see
  items 23a, potentially 23b)}
\end{itemize}
\end{minipage} & Inapplicable \\
\textbf{DISCUSSION} & & & \\
\emph{Interpretation} & 25 & \begin{minipage}[t]{\linewidth}\raggedright
\textbf{Give an overall interpretation of the main results, including
issues of fairness in the context of the objectives and previous
studies}

\begin{itemize}
\item
  \emph{Interpretation of the study results places the findings in
  context of other evidence. If there are existing models, then discuss
  the findings in the context of these existing studies}
\item
  \emph{For studies evaluating the performance of an existing prediction
  model, if existing studies have evaluated the performance of the
  model, then it's important to discuss and summarise these findings and
  place them in context}
\item
  \emph{Ensure the interpretation of the findings do not go beyond the
  findings reported from the development and evaluation of the model to
  prevent overinterpretation or `spin'}
\item
  \emph{It is useful for the reader to understand how performance of the
  model in the evaluation data compares to the performance of the model
  in any other evaluation studies of that model. When the results
  diverge, possible reasons for the difference in model performance
  should be discussed}
\end{itemize}
\end{minipage} & \\
\emph{Limitations} & 26 & \begin{minipage}[t]{\linewidth}\raggedright
\textbf{Discuss any limitations of the study (such as a
non-representative sample, sample size, overfitting, missing data) and
their effects on any biases, statistical uncertainty, and
generalizability}

\begin{itemize}
\tightlist
\item
  \emph{Acknowledgement of limitations is an important aspect of any
  scientific paper -- and can refer to any aspect of the study design,
  conduct or analysis. Provide a meaningful discussion of the study
  limitations factoring in any concerns related to representativeness of
  the data used in the analysis, sample size, overfitting and missing
  data/data quality}
\end{itemize}
\end{minipage} & \\
\emph{Usability of the model in the context of current care} & 27a &
\begin{minipage}[t]{\linewidth}\raggedright
\textbf{Describe how poor quality or unavailable input data (e.g.,
predictor values) should be assessed and handled when implementing the
prediction model}

\begin{itemize}
\item
  \emph{Authors should comment on how to handle unavailable predictor
  values at the moment the model is intended to be used as part of the
  care pathway in daily practice. Any strategies to impute missing
  values at the moment the model is intended to be used should also be
  evaluated (and thus mentioned in the Methods and Results)}
\item
  \emph{Similarly, at the point of implementation, authors should
  discuss (if relevant) the handling of poor quality input data (e.g.,
  image resolution, data format)}
\end{itemize}
\end{minipage} & \\
& 27b & \begin{minipage}[t]{\linewidth}\raggedright
\textbf{Discuss whether users will be required to interact in the
handling of the input data or use of the model, and what level of
expertise is required of users}

\begin{itemize}
\item
  \emph{Provide details on how users are expected or required to
  interact with the prediction model for the model to be used as
  intended, for example any considerations for handling the input data}
\item
  \emph{Is any expertise or training needed or required to use the
  model, handle or collect the input data, and if so, provide details}
\end{itemize}
\end{minipage} & \\
& 27c & \begin{minipage}[t]{\linewidth}\raggedright
\textbf{Discuss any next steps for future research, with a specific view
to applicability and generalizability of the model}

\begin{itemize}
\tightlist
\item
  \emph{Are further evaluations of the model needed, e.g., in different
  populations or subgroups, or is the model ready for evaluation in
  clinical trials, or implementation as part of the care pathway}
\end{itemize}
\end{minipage} & \\
\end{longtable}

\section{Introduction}\label{introduction}

\subsection{Background}\label{background}

\subsubsection{3a. Explain the healthcare context (including whether
diagnostic or prognostic) and rationale for developing or evaluating the
prediction model, including references to existing
models}\label{a.-explain-the-healthcare-context-including-whether-diagnostic-or-prognostic-and-rationale-for-developing-or-evaluating-the-prediction-model-including-references-to-existing-models}

\textbf{First paragraph:} introduction about linezolid and associated
ADR including thrombocytopenia

\textbf{Second paragraph:} what is already known in the literature about
this association (magnitude and associated factors)

\textbf{Third paragraph:} the importance of investigation this
association in Vietnamese settings and develop a risk prediction model.
Why is this study needed?

This study aimed to develop and validate a risk prediction model of
linezolid-induced thrombocytopenia adapted to Vietnamese setting. In
addition, we constructed a simplified risk score using this model to
enhance the applicability of the prediction rule in clinical practice.

\subsubsection{3b. Describe the target population and the intended
purpose of the prediction model in the context of the care pathway,
including its intended users (e.g., healthcare professionals, patients,
public)}\label{b.-describe-the-target-population-and-the-intended-purpose-of-the-prediction-model-in-the-context-of-the-care-pathway-including-its-intended-users-e.g.-healthcare-professionals-patients-public}

\subsubsection{3c. Describe any known health inequalities between
sociodemographic
groups}\label{c.-describe-any-known-health-inequalities-between-sociodemographic-groups}

\subsection{Objectives}\label{objectives}

\subsubsection{4. Specify the study objectives, including whether the
study describes the development or validation of a prediction model (or
both)}\label{specify-the-study-objectives-including-whether-the-study-describes-the-development-or-validation-of-a-prediction-model-or-both}

\section{Methods}\label{methods}

\subsection{Data}\label{data}

\subsubsection{5a. Describe the sources of data separately for the
development and evaluation datasets (e.g., randomised trial, cohort,
routine care or registry data), the rationale for using these data, and
representativeness of the
data}\label{a.-describe-the-sources-of-data-separately-for-the-development-and-evaluation-datasets-e.g.-randomised-trial-cohort-routine-care-or-registry-data-the-rationale-for-using-these-data-and-representativeness-of-the-data}

This study used data from three tertiary hospitals in Northern Vietnam:
Thanh Nhan Hospital, Bach Mai Hospital, and the National Hospital of
Tropical Diseases. Patients hospitalized and treated with linezolid were
included. The following patients were excluded: (i) those under 18 years
of age; (ii) those treated with linezolid for less than 3 days; (iii)
those without any recorded platelet count in the period before or after
initiation of linezolid therapy; (iv) those with baseline platelet count
of \textgreater{} 450 x 10\textsuperscript{9} cells/L; (v) those with
any missing recorded values among the specified predictors. Each patient
was included only once per admission and the first linezolid treatment
course was evaluated. Included patients were followed up until the end
of the linezolid treatment course or discharge date whichever comes
first.

\subsubsection{5b. Specify the dates of the collected participant data,
including start and end of participant accrual; and, if applicable, end
of
follow-up}\label{b.-specify-the-dates-of-the-collected-participant-data-including-start-and-end-of-participant-accrual-and-if-applicable-end-of-follow-up}

The data was collected from each hospital in two phases: a pilot phase
and an extension phase. In the pilot phase, we requested existing
datasets at the hospitals. In the extension phase, additional data was
collected prospectively. Data was extracted from the electronic medical
records of the hospitals, except for the pilot dataset at Bach Mai
Hospital which was extracted from physical records. In order to
harmonise different datasets, data was filled out in a paper form and
stored in Excel.

The pilot datasets were collected from January 01 to June 30, 2020 at
Thanh Nhan Hospital; from November 01 to December 31, 2019 at Bach Mai
Hospital; from May 01 to December 31, 2021 at the National Hospital of
Tropical Diseases. The extension datasets were collected from September
01, 2022 to March 31, 2023 at Thanh Nhan Hospital; from December 01,
2022 to March 31, 2023 at Bach Mai Hospital; from April 01 to September
31, 2022 at the National Hospital of Tropical Diseases. \emph{(comment:
no data of total number of patients admitted to these hospitals during
each period)}

The anonymized data were extracted from electronical medical records at
each medical institution, except data from Bach Mai Hospital in the
pilot phase. Individual ID number were assigned to each patient's
hospital admission.

Ethical approval was obtained from\ldots.

\subsection{Participants}\label{participants}

\subsubsection{6a. Specify key elements of the study setting (e.g.,
primary care, secondary care, general population) including the number
and location of
centres}\label{a.-specify-key-elements-of-the-study-setting-e.g.-primary-care-secondary-care-general-population-including-the-number-and-location-of-centres}

\subsubsection{6b. Describe the eligibility criteria for study
participants}\label{b.-describe-the-eligibility-criteria-for-study-participants}

\subsubsection{6c. Give details of any treatments received, and how they
were handled during model development or evaluation, if
relevant}\label{c.-give-details-of-any-treatments-received-and-how-they-were-handled-during-model-development-or-evaluation-if-relevant}

\subsection{Data preparation}\label{data-preparation}

\subsubsection{7. Describe any data pre-processing and quality checking,
including whether this was similar across relevant sociodemographic
groups}\label{describe-any-data-pre-processing-and-quality-checking-including-whether-this-was-similar-across-relevant-sociodemographic-groups}

\subsubsection{C-7b. Describe the method for assessing risk of bias and
applicability in the individual clusters (eg, using
PROBAST)}\label{c-7b.-describe-the-method-for-assessing-risk-of-bias-and-applicability-in-the-individual-clusters-eg-using-probast}

\subsection{Outcome}\label{outcome}

\subsubsection{8a. Clearly define the outcome that is being predicted
and the time horizon, including how and when assessed, the rationale for
choosing this outcome, and whether the method of outcome assessment is
consistent across sociodemographic
groups}\label{a.-clearly-define-the-outcome-that-is-being-predicted-and-the-time-horizon-including-how-and-when-assessed-the-rationale-for-choosing-this-outcome-and-whether-the-method-of-outcome-assessment-is-consistent-across-sociodemographic-groups}

The outcome of interest is linezolid-induced thrombocytopenia, defined
as (i) a platelet count of \textless{} 112.5 x 10\textsuperscript{9}
cells/L (75\% of the lower limit of normal) for patients with a baseline
platelet count in the normal range; (ii) A reduction in platelet count
of ≥ 25\% from the baseline value for patients with a baseline platelet
count of \textless{} 150 x 10\textsuperscript{9} cells/L {[}1--3{]}.

Normal platelet count is defined as 150-450 x 10\textsuperscript{9}
cells/L. Baseline platelet count is defined as the last recorded PLT
value before the start of linezolid therapy. Participants are considered
to have met the outcome if their platelet count value meets the above
criteria at any time during linezolid therapy or within 5 days after the
end of therapy.

\begin{tcolorbox}[enhanced jigsaw, left=2mm, breakable, rightrule=.15mm, opacitybacktitle=0.6, opacityback=0, titlerule=0mm, toptitle=1mm, toprule=.15mm, bottomrule=.15mm, leftrule=.75mm, title=\textcolor{quarto-callout-warning-color}{\faExclamationTriangle}\hspace{0.5em}{Warning}, coltitle=black, colbacktitle=quarto-callout-warning-color!10!white, colback=white, bottomtitle=1mm, arc=.35mm, colframe=quarto-callout-warning-color-frame]

Thrombocytopenia may occur within a few days after stopping LZD, when
the drug hasn't been completely eliminated. However, it is unknown
exactly how long after stopping LZD can a TP event still be attributed
to LZD use. We deemed that any TP events that occur after 5 days of
stopping LZD would not be related to LZD use.

Our rationale is that after 5 days (120 hrs), LZD is guaranteed to be
completely eliminated in all patients, as the longest
t\textsubscript{1/2} is 8.3 ± 2.4 hrs in end-stage renal disease
patients, +3 SD would be \textasciitilde16 hrs, so 120 hrs is
\textgreater7 half-lives, therefore in patients with the worst
clearance, 99\% of them would have 99\% of the drug eliminated from
their systems. Furthermore, trough LZD concentration
(C\textsubscript{min}) has previously been identified as a predictor of
LI-TP development, and LI-TP itself is mostly reversible after
discontinuation, so we would argue that any TP events that occur after
LZD has been eliminated from the system would not be related to LZD use.

\end{tcolorbox}

\subsubsection{8b. If outcome assessment requires subjective
interpretation, describe the qualifications and demographic
characteristics of the outcome
assessors}\label{b.-if-outcome-assessment-requires-subjective-interpretation-describe-the-qualifications-and-demographic-characteristics-of-the-outcome-assessors}

\subsubsection{8c. Report any actions to blind assessment of the outcome
to be
predicted}\label{c.-report-any-actions-to-blind-assessment-of-the-outcome-to-be-predicted}

\subsection{Predictors}\label{predictors}

\subsubsection{9a. Describe the choice of initial predictors (e.g.,
literature, previous models, all available predictors) and any
pre-selection of predictors before model
building}\label{a.-describe-the-choice-of-initial-predictors-e.g.-literature-previous-models-all-available-predictors-and-any-pre-selection-of-predictors-before-model-building}

\subsubsection{9b. Clearly define all predictors, including how and when
they were measured (and any actions to blind assessment of predictors
for the outcome and other
predictors)}\label{b.-clearly-define-all-predictors-including-how-and-when-they-were-measured-and-any-actions-to-blind-assessment-of-predictors-for-the-outcome-and-other-predictors}

Predictors will be screened for inclusion in the model if they meet all
of the following criteria: (i) has been identified as a risk factor of
LI-TP in previous studies; (ii) can be collected or evaluated from the
information in the datasets; (iii) for concomitant medications, has
drug-induced immune thrombocytopenia as an adverse drug reaction with a
frequency of at least \textgreater{} 1/1000 in the drug label or
Micromedex; (iv) has consensus from a clinical expert panel as possibly
related to LI-TP development.

The following information was extracted from all records:

\begin{itemize}
\tightlist
\item
  Patient demographics
\item
  Clinical department where linezolid was initiated.
\item
  Co-morbidities
\item
  Invasive procedures performed
\item
  Infection type
\item
  Laboratory results
\item
  Linezolid route of administration
\item
  Linezolid dose in milligrams.
\item
  Linezolid duration, defined as the number of days from the first to
  the last dose of linezolid.
\item
  Concomitant medications during linezolid therapy
\end{itemize}

\subsubsection{9c. If predictor measurement requires subjective
interpretation, describe the qualifications and demographic
characteristics of the predictor
assessors}\label{c.-if-predictor-measurement-requires-subjective-interpretation-describe-the-qualifications-and-demographic-characteristics-of-the-predictor-assessors}

\subsection{Sample size}\label{sample-size}

\subsubsection{10. Explain how the study size was arrived at (separately
for development and evaluation), and justify that the study size was
sufficient to answer the research question. Include details of any
sample size
calculation}\label{explain-how-the-study-size-was-arrived-at-separately-for-development-and-evaluation-and-justify-that-the-study-size-was-sufficient-to-answer-the-research-question.-include-details-of-any-sample-size-calculation}

Previous studies developing logistic regression models for LI-TP risk
predictions have included 4-6 predictors in their final models
{[}2,4--6{]}. We expect to include about as many candidate predictors,
based on results from the expert opinion survey and the Bayesian Model
Selection algorithm. Some of the candidate predictors might be
continuous, which may potentially require non-linear modelling and
therefore slightly increase the number of variables.

A general rule of thumb is for at least 10 events be available for each
candidate predictor considered in a prediction model {[}7{]}. We have a
total of 816 eligible patients and 264 of those have experienced the
outcome. If the number of candidate predictors is 7, we would have 37
events per candidate predictor, which is considerably greater than the
minimum number required. Even if the number of parameters screened is
20, we would still have 13 events per candidate predictor.

However, the aforementioned rule of thumb have generated some debate in
the literature, with recent results suggesting that event per variable
criterion is too simplistic and has no strong relation to the predictive
performance of a model. Riley et al {[}8{]} proposed a different set of
criteria to estimate minimum sample size for models developed using
logistic regression, which are the following:

\begin{enumerate}
\def\labelenumi{\arabic{enumi}.}
\tightlist
\item
  Small optimism in predictor effect estimates, defined as a global
  shrinkage factor of \textgreater= 0.9.
\item
  Small absolute difference of \textless= 0.05 in the model's apparent
  and adjusted Nagelkerke's R-squared.
\item
  Precise estimation of the overall risk in the population.
\end{enumerate}

Criteria 1 and 2 aims to reduce the potential of overfitting. Criteria 3
aims to ensure the overall risk is estimated precisely.

\paragraph{Step 1: Choose the number of candidate predictors of interest
for inclusion in the model, and calculate the corresponding number of
predictor parameters
(p)}\label{step-1-choose-the-number-of-candidate-predictors-of-interest-for-inclusion-in-the-model-and-calculate-the-corresponding-number-of-predictor-parameters-p}

Note that one predictor may require two or more parameters. For example,
a k-category predictor requires k-1 parameters and a continuous
predictor model with a non-linear trend requires more than one parameter
to be estimated. Also include any potential interaction terms towards
the total p.

When using a predictor selection method, p should be defined as the
total number of parameters screened, and not just the subset that are
included in the final model.

Assuming maximum total p to be 20.

\begin{tcolorbox}[enhanced jigsaw, left=2mm, breakable, rightrule=.15mm, opacitybacktitle=0.6, opacityback=0, titlerule=0mm, toptitle=1mm, toprule=.15mm, bottomrule=.15mm, leftrule=.75mm, title=\textcolor{quarto-callout-note-color}{\faInfo}\hspace{0.5em}{Note}, coltitle=black, colbacktitle=quarto-callout-note-color!10!white, colback=white, bottomtitle=1mm, arc=.35mm, colframe=quarto-callout-note-color-frame]

The value of p is assumed to be no larger than 20 because univariate
regression shows there are 20 variables that are significantly
correlated with the outcome.

\end{tcolorbox}

\textsubscript{Source:
\href{https://AnTangQuoc.github.io/LZD-TP-pred-model/index.qmd.html}{Article
Notebook}}

\paragraph{\texorpdfstring{Step 2: Choose sensible values for
R\textsuperscript{2}\textsubscript{CS\_adj} and
max(R\textsuperscript{2}\textsubscript{CS\_app}) based on previous
studies where R\textsuperscript{2}\textsubscript{CS} is the Cox-Snell
R\textsuperscript{2}
statistic.}{Step 2: Choose sensible values for R2CS\_adj and max(R2CS\_app) based on previous studies where R2CS is the Cox-Snell R2 statistic.}}\label{step-2-choose-sensible-values-for-r2cs_adj-and-maxr2cs_app-based-on-previous-studies-where-r2cs-is-the-cox-snell-r2-statistic.}

The value of max(R\textsuperscript{2}\textsubscript{CS\_app}) is based
on the overall prevalence or overall rate of the outcome in the
population of interest. The incidence of LI-TP in patients treated with
linezolid was estimated to be 37\% in a meta-analysis by Zhao et al
{[}9{]}.

The value of R\textsuperscript{2}\textsubscript{CS\_adj} could be based
on that for a previously published model in the same setting and
population (with similar outcome definition). However, as previous
studies does not provide any information to identify a sensible value of
the minimum expected Cox-Snell R\textsuperscript{2}, the value
R\textsuperscript{2}\textsubscript{CS\_adj} will be assumed to
correspond to a R\textsuperscript{2}\textsubscript{Nagelkerke} of 0.50,
as baseline platelet count, a ``direct'' measurement of the outcome, is
likely to be a predictor.

\begin{verbatim}
[1] 0.2048324
\end{verbatim}

\textsubscript{Source:
\href{https://AnTangQuoc.github.io/LZD-TP-pred-model/index.qmd.html}{Article
Notebook}}

\paragraph{Step 3: Criterion 1}\label{step-3-criterion-1}

Calculate the sample size required to ensure Van Houwelingen's global
shrinkage factor (S\textsubscript{VH}) is close to 1. A value of
S\textsubscript{VH} \textgreater= 0.90 is generally recommended, which
reflects a small amount of overfitting during model development.

\begin{verbatim}
[1] 775
\end{verbatim}

\begin{verbatim}
[1] 21
\end{verbatim}

\textsubscript{Source:
\href{https://AnTangQuoc.github.io/LZD-TP-pred-model/index.qmd.html}{Article
Notebook}}

We see that 775 participants are required to meet criterion 1.

\paragraph{Step 4: Criterion 2}\label{step-4-criterion-2}

Calculate the shrinkage factor (S\textsubscript{VH}) required to ensure
a small absolute difference of \textless= 0.05 in the developed model's
apparent and adjusted Nagelkerke's R\textsuperscript{2}. Then derive the
required sample size conditional on this value of S\textsubscript{VH}.

\begin{verbatim}
[1] 478
\end{verbatim}

\begin{verbatim}
[1] 34
\end{verbatim}

\textsubscript{Source:
\href{https://AnTangQuoc.github.io/LZD-TP-pred-model/index.qmd.html}{Article
Notebook}}

We see that 478 participants are required to meet criterion 2.

\paragraph{Step 5: Criterion 3}\label{step-5-criterion-3}

Calculate the sample size required to ensure a precise estimate of the
overall risk in the population. The suggested absolute margin of error
is \textless= 0.05.

\begin{verbatim}
[1] 359
\end{verbatim}

\textsubscript{Source:
\href{https://AnTangQuoc.github.io/LZD-TP-pred-model/index.qmd.html}{Article
Notebook}}

We see that 359 participants are required to meet criterion 3.

\paragraph{Step 6: Final sample size}\label{step-6-final-sample-size}

The required minimum sample size is the maximum value from steps 3 to 5,
to ensure that each of criteria 1 to 3 are met.

\begin{verbatim}
[1] 775
\end{verbatim}

\begin{verbatim}
[1] 21
\end{verbatim}

\textsubscript{Source:
\href{https://AnTangQuoc.github.io/LZD-TP-pred-model/index.qmd.html}{Article
Notebook}}

The final estimate of minimum sample size is 775, therefore our data is
sufficient for model development with 20 parameters.

The maximum number of parameters that can be screened is 21.

\subsection{Missing data}\label{missing-data}

\subsubsection{11. Describe how missing data were handled. Provide
reasons for omitting any
data}\label{describe-how-missing-data-were-handled.-provide-reasons-for-omitting-any-data}

\subsection{Analytical methods}\label{analytical-methods}

\subsubsection{12a. Describe how the data were used (e.g., for
development and evaluation of model performance) in the analysis,
including whether the data were partitioned, considering any sample size
requirements}\label{a.-describe-how-the-data-were-used-e.g.-for-development-and-evaluation-of-model-performance-in-the-analysis-including-whether-the-data-were-partitioned-considering-any-sample-size-requirements}

\subsubsection{12b. Depending on the type of model, describe how
predictors were handled in the analyses (functional form, rescaling,
transformation, or any
standardisation)}\label{b.-depending-on-the-type-of-model-describe-how-predictors-were-handled-in-the-analyses-functional-form-rescaling-transformation-or-any-standardisation}

\subsubsection{12c. Specify the type of model, rationale2, all
model-building steps, including any hyperparameter tuning, and method
for internal
validation}\label{c.-specify-the-type-of-model-rationale2-all-model-building-steps-including-any-hyperparameter-tuning-and-method-for-internal-validation}

\subsubsection{12d. Describe if and how any heterogeneity in estimates
of model parameter values and model performance was handled and
quantified across clusters (e.g., hospitals,
countries).}\label{d.-describe-if-and-how-any-heterogeneity-in-estimates-of-model-parameter-values-and-model-performance-was-handled-and-quantified-across-clusters-e.g.-hospitals-countries.}

Harmonisation between datasets was mainly done via manually recording
data to a standardized form. Data was then entered into an Excel
spreadsheet. Data cleaning was done by handling duplicates, checking for
missing values and inconsistencies. Multiple linezolid treatment
episodes in the same patient were treated as duplicates and only the
first episode was included in the analysis. Patients with missing values
were excluded from subsequent analyses. Inconsistencies were resolved by
referring back to the original records.

Before analysis, the extracted predictors are limited to those that meet
criteria (i) to (iii) in the previous section:

\begin{itemize}
\tightlist
\item
  Patient demographics were limited to age in years, gender, and weight
  in kilograms.
\item
  Clinical department was recorded into binary variables: intensive care
  unit, emergency department, and others.
\item
  Co-morbidities were recorded into binary variables: hypertension,
  heart failure, angina, myocardial infarction, cerebral vascular
  accident, diabetes, chronic obstructive pulmonary disease, cirrhosis,
  malignancies, and hematological disorders.
\item
  Invasive procedures were recorded into binary variables: endotracheal
  intubation, central venous catheter insertion, intermittent
  hemodialysis, and continuous renal replacement therapy.
\item
  Infection type was recorded into binary variables: community-acquired
  pneumonia, hospital-acquired pneumonia, skin and soft tissue
  infection, central nervous system infection, intra-abdominal
  infection, urinary tract infection, bone and joint infection,
  septicemia, and sepsis.
\item
  Laboratory results were limited to serum creatinine, hemoglobin count,
  white blood cell count, and platelet count. Creatinine clearance was
  estimated from serum creatinine using the Cockcroft-Gault equation.
\item
  Linezolid route of administration was recorded into binary variables:
  intravenous, oral, and both.
\item
  Linezolid dose in milligrams.
\item
  Linezolid duration in days.
\item
  Concomitant medications were recoded to binary variables: carbapenems,
  daptomycin, teicoplanin, levofloxacin, ibuprofen, naproxen, heparin,
  clopidogrel, enoxaparin, eptifibatide, carbamazepine, valproic acid,
  quetiapine, atezolizumab, pembrolizumab, trastuzumab, tacrolimus,
  fluorouracil, irinotecan, leucovorin, oxaliplatin, pyrazinamide, and
  rifampin.
\end{itemize}

\subsubsection{12e. Specify all measures and plots used (and their
rationale) to evaluate model performance (e.g., discrimination,
calibration, clinical utility) and, if relevant, to compare multiple
models}\label{e.-specify-all-measures-and-plots-used-and-their-rationale-to-evaluate-model-performance-e.g.-discrimination-calibration-clinical-utility-and-if-relevant-to-compare-multiple-models}

\subsubsection{12f. Describe any model updating (e.g., recalibration)
arising from the model evaluation, either overall or for particular
sociodemographic groups or
settings}\label{f.-describe-any-model-updating-e.g.-recalibration-arising-from-the-model-evaluation-either-overall-or-for-particular-sociodemographic-groups-or-settings}

Inapplicable for a development study.

\subsubsection{12g. For model evaluation, describe how the model
predictions were calculated (e.g., formula, code, object, application
programming
interface)}\label{g.-for-model-evaluation-describe-how-the-model-predictions-were-calculated-e.g.-formula-code-object-application-programming-interface}

Inapplicable for a development study.

\subsection{Class imbalance}\label{class-imbalance}

\subsubsection{13. If class imbalance methods were used, state why and
how this was done, and any subsequent methods to recalibrate the model
or the model
predictions}\label{if-class-imbalance-methods-were-used-state-why-and-how-this-was-done-and-any-subsequent-methods-to-recalibrate-the-model-or-the-model-predictions}

\subsection{Fairness}\label{fairness}

\subsubsection{14. Describe any approaches that were used to address
model fairness and their
rationale}\label{describe-any-approaches-that-were-used-to-address-model-fairness-and-their-rationale}

\subsection{Model output}\label{model-output}

\subsubsection{15. Specify the output of the prediction model (e.g.,
probabilities, classification). Provide details and rationale for any
classification and how the thresholds were
identified}\label{specify-the-output-of-the-prediction-model-e.g.-probabilities-classification.-provide-details-and-rationale-for-any-classification-and-how-the-thresholds-were-identified}

\subsection{Development versus
evaluation}\label{development-versus-evaluation}

\subsubsection{16. Identify any differences between the development and
evaluation data in healthcare setting, eligibility criteria, outcome,
and
predictors}\label{identify-any-differences-between-the-development-and-evaluation-data-in-healthcare-setting-eligibility-criteria-outcome-and-predictors}

\subsection{Ethical approval}\label{ethical-approval}

\subsubsection{17. Name the institutional research board or ethics
committee that approved the study and describe the participant-informed
consent or the ethics committee waiver of informed
consent}\label{name-the-institutional-research-board-or-ethics-committee-that-approved-the-study-and-describe-the-participant-informed-consent-or-the-ethics-committee-waiver-of-informed-consent}

\section{Open Science}\label{open-science}

\subsection{Funding}\label{funding}

\subsubsection{18a. Give the source of funding and the role of the
funders for the present
study}\label{a.-give-the-source-of-funding-and-the-role-of-the-funders-for-the-present-study}

\subsection{Conflicts of interest}\label{conflicts-of-interest}

\subsubsection{18b. Declare any conflicts of interest and financial
disclosures for all
authors}\label{b.-declare-any-conflicts-of-interest-and-financial-disclosures-for-all-authors}

\subsection{Protocol}\label{protocol}

\subsubsection{18c. Indicate where the study protocol can be accessed or
state that a protocol was not
prepared}\label{c.-indicate-where-the-study-protocol-can-be-accessed-or-state-that-a-protocol-was-not-prepared}

\subsection{Registration}\label{registration}

\subsubsection{18d. Provide registration information for the study,
including register name and registration number, or state that the study
was not
registered}\label{d.-provide-registration-information-for-the-study-including-register-name-and-registration-number-or-state-that-the-study-was-not-registered}

\subsection{Data sharing}\label{data-sharing}

\subsubsection{18e. Provide details of the availability of the study
data}\label{e.-provide-details-of-the-availability-of-the-study-data}

\subsection{Code sharing}\label{code-sharing}

\subsubsection{18f. Provide details of the availability of the
analytical
code}\label{f.-provide-details-of-the-availability-of-the-analytical-code}

\section{Patient \& Public
Involvement}\label{patient-public-involvement}

\subsection{Patient \& Public
Involvement}\label{patient-public-involvement-1}

\subsubsection{19. Provide details of any patient and public involvement
during the design, conduct, reporting, interpretation, or dissemination
of the study or state no
involvement}\label{provide-details-of-any-patient-and-public-involvement-during-the-design-conduct-reporting-interpretation-or-dissemination-of-the-study-or-state-no-involvement}

\section{Results}\label{results}

\subsection{Participants}\label{participants-1}

\subsubsection{20a. Describe the flow of participants through the study,
including the number of participants with and without the outcome and,
if applicable, a summary of the follow-up time. A diagram may be
helpful}\label{a.-describe-the-flow-of-participants-through-the-study-including-the-number-of-participants-with-and-without-the-outcome-and-if-applicable-a-summary-of-the-follow-up-time.-a-diagram-may-be-helpful}

\subsubsection{20b. Report the characteristics overall and, where
applicable, for each data source or setting, including the key dates,
key predictors (including demographics), treatments received, sample
size, number of outcome events, follow-up time, and amount of missing
data. A table may be helpful. Report any differences across key
demographic
groups}\label{b.-report-the-characteristics-overall-and-where-applicable-for-each-data-source-or-setting-including-the-key-dates-key-predictors-including-demographics-treatments-received-sample-size-number-of-outcome-events-follow-up-time-and-amount-of-missing-data.-a-table-may-be-helpful.-report-any-differences-across-key-demographic-groups}

\subsubsection{20c. For model evaluation, show a comparison with the
development data of the distribution of important predictors
(demographics, predictors, and
outcome).}\label{c.-for-model-evaluation-show-a-comparison-with-the-development-data-of-the-distribution-of-important-predictors-demographics-predictors-and-outcome.}

Inapplicable for this study.

\subsection{Risk of bias}\label{risk-of-bias}

\subsubsection{C-11. Report the results of the risk-of-bias assessment
in the individual
clusters}\label{c-11.-report-the-results-of-the-risk-of-bias-assessment-in-the-individual-clusters}

\subsection{Model development}\label{model-development}

\subsubsection{21. Specify the number of participants and outcome events
in each analysis (e.g., for model development, hyperparameter tuning,
model
evaluation)}\label{specify-the-number-of-participants-and-outcome-events-in-each-analysis-e.g.-for-model-development-hyperparameter-tuning-model-evaluation}

\subsection{Model specification}\label{model-specification}

\subsubsection{22. Provide details of the full prediction model (e.g.,
formula, code, object, API) to allow predictions in new individuals and
to enable third-party evaluation and implementation, including any
restrictions to access or re-use (e.g., freely available,
proprietary)}\label{provide-details-of-the-full-prediction-model-e.g.-formula-code-object-api-to-allow-predictions-in-new-individuals-and-to-enable-third-party-evaluation-and-implementation-including-any-restrictions-to-access-or-re-use-e.g.-freely-available-proprietary}

\subsection{Model performance}\label{model-performance}

\subsubsection{23a. Report model performance estimates with confidence
intervals, including for any key subgroups (e.g., sociodemographic).
Consider plots to aid
presentation}\label{a.-report-model-performance-estimates-with-confidence-intervals-including-for-any-key-subgroups-e.g.-sociodemographic.-consider-plots-to-aid-presentation}

\subsubsection{23b. If examined, report results of any heterogeneity in
model performance across clusters. See TRIPOD Cluster for additional
details}\label{b.-if-examined-report-results-of-any-heterogeneity-in-model-performance-across-clusters.-see-tripod-cluster-for-additional-details}

\subsection{Model updating}\label{model-updating}

\subsubsection{24. Report the results from any model updating, including
the updated model and subsequent
performance}\label{report-the-results-from-any-model-updating-including-the-updated-model-and-subsequent-performance}

Inapplicable for development studies.

\section{Discussion}\label{discussion}

\subsection{Interpretation}\label{interpretation}

\subsubsection{25. Give an overall interpretation of the main results,
including issues of fairness in the context of the objectives and
previous
studies}\label{give-an-overall-interpretation-of-the-main-results-including-issues-of-fairness-in-the-context-of-the-objectives-and-previous-studies}

\subsection{Limitations}\label{limitations}

\subsubsection{26. Discuss any limitations of the study (such as a
non-representative sample, sample size, overfitting, missing data) and
their effects on any biases, statistical uncertainty, and
generalizability}\label{discuss-any-limitations-of-the-study-such-as-a-non-representative-sample-sample-size-overfitting-missing-data-and-their-effects-on-any-biases-statistical-uncertainty-and-generalizability}

\subsection{Usability of the model in the context of current
care}\label{usability-of-the-model-in-the-context-of-current-care}

\subsubsection{27a. Describe how poor quality or unavailable input data
(e.g., predictor values) should be assessed and handled when
implementing the prediction
model}\label{a.-describe-how-poor-quality-or-unavailable-input-data-e.g.-predictor-values-should-be-assessed-and-handled-when-implementing-the-prediction-model}

\subsubsection{27b. Discuss whether users will be required to interact
in the handling of the input data or use of the model, and what level of
expertise is required of
users}\label{b.-discuss-whether-users-will-be-required-to-interact-in-the-handling-of-the-input-data-or-use-of-the-model-and-what-level-of-expertise-is-required-of-users}

\subsubsection{27c. Discuss any next steps for future research, with a
specific view to applicability and generalizability of the
model}\label{c.-discuss-any-next-steps-for-future-research-with-a-specific-view-to-applicability-and-generalizability-of-the-model}

\section{References}\label{references}

\phantomsection\label{refs}
\begin{CSLReferences}{0}{1}
\bibitem[\citeproctext]{ref-zyvoxpr}
1. Zyvox prescribing information {[}Internet{]}. Available from:
\url{https://labeling.pfizer.com/showlabeling.aspx?id=649}

\bibitem[\citeproctext]{ref-xu_establishment_2023}
2. Xu J, Lu J, Yuan Y, Duan L, Shi L, Chen F, et al.
\href{https://doi.org/10.1093/jac/dkad191}{Establishment and validation
of a risk prediction model incorporating concentrations of linezolid and
its metabolite {PNU142300} for linezolid-induced thrombocytopenia}. The
Journal of Antimicrobial Chemotherapy. 2023;78:1974--81.

\bibitem[\citeproctext]{ref-kawasuji_proposal_2021}
3. Kawasuji H, Tsuji Y, Ogami C, Kimoto K, Ueno A, Miyajima Y, et al.
Proposal of initial and maintenance dosing regimens with linezolid for
renal impairment patients. BMC Pharmacology and Toxicology
{[}Internet{]}. 2021 {[}cited 2024 Feb 26{]};22:13. Available from:
\url{https://doi.org/10.1186/s40360-021-00479-w}

\bibitem[\citeproctext]{ref-liu_analysis_2021}
4. Liu Y, Liu T, Wei G, Yan P, Fang X, Xie L. Analysis of risk factors
and establishment of risk prediction model for linezolid-associated
thrombocytopenia. Medical Journal of Chinese People's Liberation Army
{[}Internet{]}. 2021;46. Available from:
\url{https://d.wanfangdata.com.cn/periodical/jfjyxzz202108006}

\bibitem[\citeproctext]{ref-duan_regression_2022}
5. Duan L, Zhou Q, Feng Z, Zhu C, Cai Y, Wang S, et al. A {Regression}
{Model} to {Predict} {Linezolid} {Induced} {Thrombocytopenia} in
{Neonatal} {Sepsis} {Patients}: {A} {Ten}-{Year} {Retrospective}
{Cohort} {Study}. Front Pharmacol {[}Internet{]}. 2022;13:710099.
Available from: \url{https://www.ncbi.nlm.nih.gov/pubmed/35185555}

\bibitem[\citeproctext]{ref-qin_development_2021}
6. Qin Y, Chen Z, Gao S, Pan MK, Li YX, Lv ZQ, et al. Development and
{Validation} of a {Risk} {Prediction} {Model} of {Linezolid}-induced
{Thrombocytopenia} in {Elderly} {Patients} {[}Internet{]}. In Review;
2021 Jun. Available from:
\url{https://www.researchsquare.com/article/rs-582799/v1}

\bibitem[\citeproctext]{ref-peduzzi1995}
7. Peduzzi P, Concato J, Feinstein AR, Holford TR. Importance of events
per independent variable in proportional hazards regression analysis II.
Accuracy and precision of regression estimates. Journal of Clinical
Epidemiology {[}Internet{]}. 1995;48:1503--10. Available from:
\url{https://www.jclinepi.com/article/0895-4356(95)00048-8/abstract}

\bibitem[\citeproctext]{ref-riley2019}
8. Riley RD, Snell KI, Ensor J, Burke DL, Harrell Jr FE, Moons KG, et
al. Minimum sample size for developing a multivariable prediction model:
PART II - binary and time-to-event outcomes. Statistics in Medicine
{[}Internet{]}. 2019;38:1276--96. Available from:
\url{https://onlinelibrary.wiley.com/doi/abs/10.1002/sim.7992}

\bibitem[\citeproctext]{ref-zhao_prediction_2024}
9. Zhao X, Peng Q, Hu D, Li W, Ji Q, Dong Q, et al. Prediction of risk
factors for linezolid-induced thrombocytopenia based on neural network
model. Frontiers in Pharmacology {[}Internet{]}. 2024 {[}cited 2024 Feb
27{]};15. Available from:
\url{https://www.frontiersin.org/journals/pharmacology/articles/10.3389/fphar.2024.1292828}

\end{CSLReferences}



\end{document}
