% Options for packages loaded elsewhere
\PassOptionsToPackage{unicode}{hyperref}
\PassOptionsToPackage{hyphens}{url}
\PassOptionsToPackage{dvipsnames,svgnames,x11names}{xcolor}
%
\documentclass[
  letterpaper,
  DIV=11,
  numbers=noendperiod]{scrartcl}

\usepackage{amsmath,amssymb}
\usepackage{iftex}
\ifPDFTeX
  \usepackage[T1]{fontenc}
  \usepackage[utf8]{inputenc}
  \usepackage{textcomp} % provide euro and other symbols
\else % if luatex or xetex
  \usepackage{unicode-math}
  \defaultfontfeatures{Scale=MatchLowercase}
  \defaultfontfeatures[\rmfamily]{Ligatures=TeX,Scale=1}
\fi
\usepackage{lmodern}
\ifPDFTeX\else  
    % xetex/luatex font selection
\fi
% Use upquote if available, for straight quotes in verbatim environments
\IfFileExists{upquote.sty}{\usepackage{upquote}}{}
\IfFileExists{microtype.sty}{% use microtype if available
  \usepackage[]{microtype}
  \UseMicrotypeSet[protrusion]{basicmath} % disable protrusion for tt fonts
}{}
\makeatletter
\@ifundefined{KOMAClassName}{% if non-KOMA class
  \IfFileExists{parskip.sty}{%
    \usepackage{parskip}
  }{% else
    \setlength{\parindent}{0pt}
    \setlength{\parskip}{6pt plus 2pt minus 1pt}}
}{% if KOMA class
  \KOMAoptions{parskip=half}}
\makeatother
\usepackage{xcolor}
\setlength{\emergencystretch}{3em} % prevent overfull lines
\setcounter{secnumdepth}{5}
% Make \paragraph and \subparagraph free-standing
\ifx\paragraph\undefined\else
  \let\oldparagraph\paragraph
  \renewcommand{\paragraph}[1]{\oldparagraph{#1}\mbox{}}
\fi
\ifx\subparagraph\undefined\else
  \let\oldsubparagraph\subparagraph
  \renewcommand{\subparagraph}[1]{\oldsubparagraph{#1}\mbox{}}
\fi


\providecommand{\tightlist}{%
  \setlength{\itemsep}{0pt}\setlength{\parskip}{0pt}}\usepackage{longtable,booktabs,array}
\usepackage{calc} % for calculating minipage widths
% Correct order of tables after \paragraph or \subparagraph
\usepackage{etoolbox}
\makeatletter
\patchcmd\longtable{\par}{\if@noskipsec\mbox{}\fi\par}{}{}
\makeatother
% Allow footnotes in longtable head/foot
\IfFileExists{footnotehyper.sty}{\usepackage{footnotehyper}}{\usepackage{footnote}}
\makesavenoteenv{longtable}
\usepackage{graphicx}
\makeatletter
\def\maxwidth{\ifdim\Gin@nat@width>\linewidth\linewidth\else\Gin@nat@width\fi}
\def\maxheight{\ifdim\Gin@nat@height>\textheight\textheight\else\Gin@nat@height\fi}
\makeatother
% Scale images if necessary, so that they will not overflow the page
% margins by default, and it is still possible to overwrite the defaults
% using explicit options in \includegraphics[width, height, ...]{}
\setkeys{Gin}{width=\maxwidth,height=\maxheight,keepaspectratio}
% Set default figure placement to htbp
\makeatletter
\def\fps@figure{htbp}
\makeatother
% definitions for citeproc citations
\NewDocumentCommand\citeproctext{}{}
\NewDocumentCommand\citeproc{mm}{%
  \begingroup\def\citeproctext{#2}\cite{#1}\endgroup}
\makeatletter
 % allow citations to break across lines
 \let\@cite@ofmt\@firstofone
 % avoid brackets around text for \cite:
 \def\@biblabel#1{}
 \def\@cite#1#2{{#1\if@tempswa , #2\fi}}
\makeatother
\newlength{\cslhangindent}
\setlength{\cslhangindent}{1.5em}
\newlength{\csllabelwidth}
\setlength{\csllabelwidth}{3em}
\newenvironment{CSLReferences}[2] % #1 hanging-indent, #2 entry-spacing
 {\begin{list}{}{%
  \setlength{\itemindent}{0pt}
  \setlength{\leftmargin}{0pt}
  \setlength{\parsep}{0pt}
  % turn on hanging indent if param 1 is 1
  \ifodd #1
   \setlength{\leftmargin}{\cslhangindent}
   \setlength{\itemindent}{-1\cslhangindent}
  \fi
  % set entry spacing
  \setlength{\itemsep}{#2\baselineskip}}}
 {\end{list}}
\usepackage{calc}
\newcommand{\CSLBlock}[1]{\hfill\break\parbox[t]{\linewidth}{\strut\ignorespaces#1\strut}}
\newcommand{\CSLLeftMargin}[1]{\parbox[t]{\csllabelwidth}{\strut#1\strut}}
\newcommand{\CSLRightInline}[1]{\parbox[t]{\linewidth - \csllabelwidth}{\strut#1\strut}}
\newcommand{\CSLIndent}[1]{\hspace{\cslhangindent}#1}

\KOMAoption{captions}{tableheading}
\makeatletter
\@ifpackageloaded{tcolorbox}{}{\usepackage[skins,breakable]{tcolorbox}}
\@ifpackageloaded{fontawesome5}{}{\usepackage{fontawesome5}}
\definecolor{quarto-callout-color}{HTML}{909090}
\definecolor{quarto-callout-note-color}{HTML}{0758E5}
\definecolor{quarto-callout-important-color}{HTML}{CC1914}
\definecolor{quarto-callout-warning-color}{HTML}{EB9113}
\definecolor{quarto-callout-tip-color}{HTML}{00A047}
\definecolor{quarto-callout-caution-color}{HTML}{FC5300}
\definecolor{quarto-callout-color-frame}{HTML}{acacac}
\definecolor{quarto-callout-note-color-frame}{HTML}{4582ec}
\definecolor{quarto-callout-important-color-frame}{HTML}{d9534f}
\definecolor{quarto-callout-warning-color-frame}{HTML}{f0ad4e}
\definecolor{quarto-callout-tip-color-frame}{HTML}{02b875}
\definecolor{quarto-callout-caution-color-frame}{HTML}{fd7e14}
\makeatother
\makeatletter
\@ifpackageloaded{caption}{}{\usepackage{caption}}
\AtBeginDocument{%
\ifdefined\contentsname
  \renewcommand*\contentsname{Table of contents}
\else
  \newcommand\contentsname{Table of contents}
\fi
\ifdefined\listfigurename
  \renewcommand*\listfigurename{List of Figures}
\else
  \newcommand\listfigurename{List of Figures}
\fi
\ifdefined\listtablename
  \renewcommand*\listtablename{List of Tables}
\else
  \newcommand\listtablename{List of Tables}
\fi
\ifdefined\figurename
  \renewcommand*\figurename{Figure}
\else
  \newcommand\figurename{Figure}
\fi
\ifdefined\tablename
  \renewcommand*\tablename{Table}
\else
  \newcommand\tablename{Table}
\fi
}
\@ifpackageloaded{float}{}{\usepackage{float}}
\floatstyle{ruled}
\@ifundefined{c@chapter}{\newfloat{codelisting}{h}{lop}}{\newfloat{codelisting}{h}{lop}[chapter]}
\floatname{codelisting}{Listing}
\newcommand*\listoflistings{\listof{codelisting}{List of Listings}}
\makeatother
\makeatletter
\makeatother
\makeatletter
\@ifpackageloaded{caption}{}{\usepackage{caption}}
\@ifpackageloaded{subcaption}{}{\usepackage{subcaption}}
\makeatother
\ifLuaTeX
  \usepackage{selnolig}  % disable illegal ligatures
\fi
\usepackage{bookmark}

\IfFileExists{xurl.sty}{\usepackage{xurl}}{} % add URL line breaks if available
\urlstyle{same} % disable monospaced font for URLs
\hypersetup{
  pdftitle={Development and Validation of a Risk Prediction Model of linezolid-induced thrombocytopenia},
  colorlinks=true,
  linkcolor={blue},
  filecolor={Maroon},
  citecolor={Blue},
  urlcolor={Blue},
  pdfcreator={LaTeX via pandoc}}

\title{Development and Validation of a Risk Prediction Model of
linezolid-induced thrombocytopenia}
\author{Nhi Nguyen Ha \and An Tang Quoc \and Ha Tran Ngan \and Hang
Nguyen Thi Thu \and Hoa Vu Dinh \and Nhung TH Trinh \and Anh Nguyen
Hoang}
\date{Tuesday, April 23, 2024}

\begin{document}
\maketitle
\begin{abstract}
Write abstract here, note the indentation
\end{abstract}

\section{Checklist}\label{checklist}

\begin{longtable}[]{@{}
  >{\raggedright\arraybackslash}p{(\columnwidth - 6\tabcolsep) * \real{0.2466}}
  >{\centering\arraybackslash}p{(\columnwidth - 6\tabcolsep) * \real{0.2466}}
  >{\raggedright\arraybackslash}p{(\columnwidth - 6\tabcolsep) * \real{0.2603}}
  >{\centering\arraybackslash}p{(\columnwidth - 6\tabcolsep) * \real{0.2466}}@{}}
\caption{TRIPOD-Cluster checklist of items to include when reporting a
study developing or validating a multivariable prediction model using
clustered data}\tabularnewline
\toprule\noalign{}
\begin{minipage}[b]{\linewidth}\raggedright
\textbf{Section/topic}
\end{minipage} & \begin{minipage}[b]{\linewidth}\centering
\textbf{Item No}
\end{minipage} & \begin{minipage}[b]{\linewidth}\raggedright
\textbf{Description}
\end{minipage} & \begin{minipage}[b]{\linewidth}\centering
\textbf{Draft date}
\end{minipage} \\
\midrule\noalign{}
\endfirsthead
\toprule\noalign{}
\begin{minipage}[b]{\linewidth}\raggedright
\textbf{Section/topic}
\end{minipage} & \begin{minipage}[b]{\linewidth}\centering
\textbf{Item No}
\end{minipage} & \begin{minipage}[b]{\linewidth}\raggedright
\textbf{Description}
\end{minipage} & \begin{minipage}[b]{\linewidth}\centering
\textbf{Draft date}
\end{minipage} \\
\midrule\noalign{}
\endhead
\bottomrule\noalign{}
\endlastfoot
\textbf{Title and abstract} & & & \\
Title & 1 & Identify the study as developing and/or validating a
multivariable prediction model, the target population, and the outcome
to be predicted & \\
Abstract & 2 & Provide a summary of research objectives, setting,
participants, data source, sample size, predictors, outcome, statistical
analysis, results, and conclusions* & \\
\textbf{Introduction} & & & \\
Background and objectives & 3a & Explain the medical context (including
whether diagnostic or prognostic) and rationale for developing or
validating the prediction model, including references to existing
models, and the advantages of the study design* & \\
& 3b & Specify the objectives, including whether the study describes the
development or validation of the model* & \\
\textbf{Methods} & & & \\
Participants and data & 4a & Describe eligibility criteria for
participants and datasets* & \\
& 4b & Describe the origin of the data, and how the data were
identified, requested, and collected & \\
Sample size & 5 & Explain how the sample size was arrived at* & Mar
21 \\
Outcomes and predictors & 6a & Define the outcome that is predicted by
the model, including how and when assessed* & Mar 21 \\
& 6b & Define all predictors used in developing or validating the model,
including how and when measured* & \\
Data preparation & 7a & Describe how the data were prepared for
analysis, including any cleaning, harmonisation, linkage, and quality
checks & \\
& 7b & Describe the method for assessing risk of bias and applicability
in the individual clusters (eg, using PROBAST) & \\
& 7c & For validation, identify any differences in definition and
measurement from the development data (eg, setting, eligibility
criteria, outcome, predictors)* & \\
& 7d & Describe how missing data were handled* & \\
Data analysis & 8a & Describe how predictors were handled in the
analyses & \\
& 8b & Specify the type of model, all model building procedures (eg, any
predictor selection and penalisation), and method for validation* & \\
& 8c & Describe how any heterogeneity across clusters (eg, studies or
settings) in model parameter values was handled & \\
& 8d & For validation, describe how the predictions were calculated & \\
& 8e & Specify all measures used to assess model performance (eg,
calibration, discrimination, and decision curve analysis) and, if
relevant, to compare multiple models & \\
& 8f & Describe how any heterogeneity across clusters (eg, studies or
settings) in model performance was handled and quantified & \\
& 8g & Describe any model updating (eg, recalibration) arising from the
validation, either overall or for particular populations or settings*
& \\
Sensitivity analysis & 9 & Describe any planned subgroup or sensitivity
analysis---eg, assessing performance according to sources of bias,
participant characteristics, setting & \\
\textbf{Results} & & & \\
Participants and datasets & 10a & Describe the number of clusters and
participants from data identified through to data analysed; a flowchart
might be helpful* & \\
& 10b & Report the characteristics overall and where applicable for each
data source or setting, including the key dates, predictors, treatments
received, sample size, number of outcome events, follow-up time, and
amount of missing data* & \\
& 10c & For validation, show a comparison with the development data of
the distribution of important variables (demographics, predictors, and
outcome) & \\
Risk of bias & 11 & Report the results of the risk-of-bias assessment in
the individual clusters & \\
Model development and specification & 12a & Report the results of any
assessments of heterogeneity across clusters that led to subsequent
actions during the model's development (eg, inclusion or exclusion of
particular predictors or clusters) & \\
& 12b & Present the final prediction model (ie, all regression
coefficients, and model intercept or baseline estimate of the outcome at
a given time point) and explain how to use it for predictions in new
individuals* & \\
Model performance & 13a & Report performance measures (with uncertainty
intervals) for the prediction model, overall and for each cluster & \\
& 13b & Report results of any heterogeneity across clusters in model
performance & \\
Model updating & 14 & Report the results from any model updating
(including the updated model equation and subsequent performance),
overall and for each cluster* & \\
Sensitivity analysis & 15 & Report results from any subgroup or
sensitivity analysis & \\
\textbf{Discussion} & & & \\
Interpretation & 16a & Give an overall interpretation of the main
results, including heterogeneity across clusters in model performance,
in the context of the objectives and previous studies* & \\
& 16b & For validation, discuss the results with reference to the model
performance in the development data, and in any previous validations
& \\
& 16c & Discuss the strengths of the study and any limitations (eg,
missing or incomplete data, non-representativeness, data harmonisation
problems) & \\
Implications & 17 & Discuss the potential use of the model and
implications for future research, with specific view to generalisability
and applicability of the model across different settings or
(sub)populations & \\
\textbf{Other information} & & & \\
Supplementary information & 18 & Provide information about the
availability of supplementary resources (eg, study protocol, analysis
code, datasets)* & \\
Funding & 19 & Give the source of funding and the role of the funders
for the present study & \\
\end{longtable}

\section{Introduction}\label{introduction}

\subsection{Background and objectives}\label{background-and-objectives}

\textbf{First paragraph:} introduction about linezolid and associated
ADR including thrombocytopenia

\textbf{Second paragraph:} what is already known in the literature about
this association (magnitude and associated factors)

\textbf{Third paragraph:} the importance of investigation this
association in Vietnamese settings and develop a risk prediction model.
Why is this study needed?

This study aimed to develop and validate a risk prediction model of
linezolid-induced thrombocytopenia adapted to Vietnamese setting. In
addition, we constructed a simplified risk score using this model to
enhance the applicability of the prediction rule in clinical practice.

\section{Methods}\label{methods}

\subsection{Participants and data}\label{participants-and-data}

\subsubsection{4a: Describe eligibility criteria for participants and
datasets}\label{a-describe-eligibility-criteria-for-participants-and-datasets}

This study used data from three tertiary hospitals in Northern Vietnam:
Thanh Nhan Hospital, Bach Mai Hospital, and the National Hospital of
Tropical Diseases. Patients hospitalized and treated with linezolid were
included. The following patients were excluded: (i) those under 18 years
of age; (ii) those treated with linezolid for less than 3 days; (iii)
those without any recorded platelet count in the period before or after
initiation of linezolid therapy; (iv) those with baseline platelet count
of \textgreater{} 450 x 10\textsuperscript{9} cells/L; (v) those with
any missing recorded values among the specified predictors. Each patient
was included only once per admission and the first linezolid treatment
course was evaluated. Included patients were followed up until the end
of the linezolid treatment course or discharge date whichever comes
first.

\subsubsection{4b: Describe the origin of the data, and how the data
were identified, requested, and
collected}\label{b-describe-the-origin-of-the-data-and-how-the-data-were-identified-requested-and-collected}

The data was collected from each hospital in two phases: a pilot phase
and an extension phase. In the pilot phase, we requested existing
datasets at the hospitals. In the extension phase, additional data was
collected prospectively. Data was extracted from the electronic medical
records of the hospitals, except for the pilot dataset at Bach Mai
Hospital which was extracted from physical records. In order to
harmonise different datasets, data was filled out in a paper form and
stored in Excel.

The pilot datasets were collected from January 01 to June 30, 2020 at
Thanh Nhan Hospital; from November 01 to December 31, 2019 at Bach Mai
Hospital; from May 01 to December 31, 2021 at the National Hospital of
Tropical Diseases. The extension datasets were collected from September
01, 2022 to March 31, 2023 at Thanh Nhan Hospital; from December 01,
2022 to March 31, 2023 at Bach Mai Hospital; from April 01 to September
31, 2022 at the National Hospital of Tropical Diseases. \emph{(comment:
no data of total number of patients admitted to these hospitals during
each period)}

The anonymized data were extracted from electronical medical records at
each medical institution, except data from Bach Mai Hospital in the
pilot phase. Individual ID number were assigned to each patient's
hospital admission.

Ethical approval was obtained from\ldots.

\subsection{Sample size}\label{sample-size}

Previous studies developing logistic regression models for LI-TP risk
predictions have included 4-6 predictors in their final models
{[}1--4{]}. We expect to include about as many candidate predictors,
based on results from the expert opinion survey and the Bayesian Model
Selection algorithm see~\ref{sec-outcomes-and-predictors}. Some of the
candidate predictors might be continuous, which may potentially require
non-linear modelling and therefore slightly increase the number of
variables.

A general rule of thumb is for at least 10 events be available for each
candidate predictor considered in a prediction model {[}5{]}. We have a
total of 816 eligible patients and 264 of those have experienced the
outcome. If the number of candidate predictors is 7, we would have 37
events per candidate predictor, which is considerably greater than the
minimum number required. Even if the number of parameters screened is
20, we would still have 13 events per candidate predictor.

However, the aforementioned rule of thumb have generated some debate in
the literature, with recent results suggesting that event per variable
criterion is too simplistic and has no strong relation to the predictive
performance of a model. Riley et al {[}6{]} proposed a different set of
criteria to estimate minimum sample size for models developed using
logistic regression, which are the following:

\begin{enumerate}
\def\labelenumi{\arabic{enumi}.}
\tightlist
\item
  Small optimism in predictor effect estimates, defined as a global
  shrinkage factor of \textgreater= 0.9.
\item
  Small absolute difference of \textless= 0.05 in the model's apparent
  and adjusted Nagelkerke's R-squared.
\item
  Precise estimation of the overall risk in the population.
\end{enumerate}

Criteria 1 and 2 aims to reduce the potential of overfitting. Criteria 3
aims to ensure the overall risk is estimated precisely.

\subsubsection{Step 1: Choose the number of candidate predictors of
interest for inclusion in the model, and calculate the corresponding
number of predictor parameters
(p)}\label{step-1-choose-the-number-of-candidate-predictors-of-interest-for-inclusion-in-the-model-and-calculate-the-corresponding-number-of-predictor-parameters-p}

Note that one predictor may require two or more parameters. For example,
a k-category predictor requires k-1 parameters and a continuous
predictor model with a non-linear trend requires more than one parameter
to be estimated. Also include any potential interaction terms towards
the total p.

When using a predictor selection method, p should be defined as the
total number of parameters screened, and not just the subset that are
included in the final model.

Assuming maximum total p to be 20.

\begin{tcolorbox}[enhanced jigsaw, breakable, opacitybacktitle=0.6, leftrule=.75mm, opacityback=0, title=\textcolor{quarto-callout-note-color}{\faInfo}\hspace{0.5em}{Note}, coltitle=black, bottomtitle=1mm, toprule=.15mm, toptitle=1mm, colback=white, titlerule=0mm, colframe=quarto-callout-note-color-frame, arc=.35mm, bottomrule=.15mm, rightrule=.15mm, left=2mm, colbacktitle=quarto-callout-note-color!10!white]

The value of p is assumed to be no larger than 20 because univariate
regression shows there are 20 variables that are significantly
correlated with the outcome.

\end{tcolorbox}

\textsubscript{Source:
\href{https://AnTangQuoc.github.io/LZD-TP-pred-model/index.qmd.html}{Article
Notebook}}

\subsubsection{\texorpdfstring{Step 2: Choose sensible values for
R\textsuperscript{2}\textsubscript{CS\_adj} and
max(R\textsuperscript{2}\textsubscript{CS\_app}) based on previous
studies where R\textsuperscript{2}\textsubscript{CS} is the Cox-Snell
R\textsuperscript{2}
statistic.}{Step 2: Choose sensible values for R2CS\_adj and max(R2CS\_app) based on previous studies where R2CS is the Cox-Snell R2 statistic.}}\label{step-2-choose-sensible-values-for-r2cs_adj-and-maxr2cs_app-based-on-previous-studies-where-r2cs-is-the-cox-snell-r2-statistic.}

The value of max(R\textsuperscript{2}\textsubscript{CS\_app}) is based
on the overall prevalence or overall rate of the outcome in the
population of interest. The incidence of LI-TP in patients treated with
linezolid was estimated to be 37\% in a meta-analysis by Zhao et al
{[}7{]}.

The value of R\textsuperscript{2}\textsubscript{CS\_adj} could be based
on that for a previously published model in the same setting and
population (with similar outcome definition). However, as previous
studies does not provide any information to identify a sensible value of
the minimum expected Cox-Snell R\textsuperscript{2}, the value
R\textsuperscript{2}\textsubscript{CS\_adj} will be assumed to
correspond to a R\textsuperscript{2}\textsubscript{Nagelkerke} of 0.50,
as baseline platelet count, a ``direct'' measurement of the outcome, is
likely to be a predictor.

\begin{verbatim}
[1] 0.2048324
\end{verbatim}

\textsubscript{Source:
\href{https://AnTangQuoc.github.io/LZD-TP-pred-model/index.qmd.html}{Article
Notebook}}

\subsubsection{Step 3: Criterion 1}\label{step-3-criterion-1}

Calculate the sample size required to ensure Van Houwelingen's global
shrinkage factor (S\textsubscript{VH}) is close to 1. A value of
S\textsubscript{VH} \textgreater= 0.90 is generally recommended, which
reflects a small amount of overfitting during model development.

\begin{verbatim}
[1] 775
\end{verbatim}

\begin{verbatim}
[1] 21
\end{verbatim}

\textsubscript{Source:
\href{https://AnTangQuoc.github.io/LZD-TP-pred-model/index.qmd.html}{Article
Notebook}}

We see that 775 participants are required to meet criterion 1.

\subsubsection{Step 4: Criterion 2}\label{step-4-criterion-2}

Calculate the shrinkage factor (S\textsubscript{VH}) required to ensure
a small absolute difference of \textless= 0.05 in the developed model's
apparent and adjusted Nagelkerke's R\textsuperscript{2}. Then derive the
required sample size conditional on this value of S\textsubscript{VH}.

\begin{verbatim}
[1] 478
\end{verbatim}

\begin{verbatim}
[1] 34
\end{verbatim}

\textsubscript{Source:
\href{https://AnTangQuoc.github.io/LZD-TP-pred-model/index.qmd.html}{Article
Notebook}}

We see that 478 participants are required to meet criterion 2.

\subsubsection{Step 5: Criterion 3}\label{step-5-criterion-3}

Calculate the sample size required to ensure a precise estimate of the
overall risk in the population. The suggested absolute margin of error
is \textless= 0.05.

\begin{verbatim}
[1] 359
\end{verbatim}

\textsubscript{Source:
\href{https://AnTangQuoc.github.io/LZD-TP-pred-model/index.qmd.html}{Article
Notebook}}

We see that 359 participants are required to meet criterion 3.

\subsubsection{Step 6: Final sample
size}\label{step-6-final-sample-size}

The required minimum sample size is the maximum value from steps 3 to 5,
to ensure that each of criteria 1 to 3 are met.

\begin{verbatim}
[1] 775
\end{verbatim}

\begin{verbatim}
[1] 21
\end{verbatim}

\textsubscript{Source:
\href{https://AnTangQuoc.github.io/LZD-TP-pred-model/index.qmd.html}{Article
Notebook}}

The final estimate of minimum sample size is 775, therefore our data is
sufficient for model development with 20 parameters.

The maximum number of parameters that can be screened is 21.

\subsection{Outcomes and predictors}\label{sec-outcomes-and-predictors}

\subsubsection{6a. Define the outcome that is predicted by the model,
including how and when
assessed}\label{a.-define-the-outcome-that-is-predicted-by-the-model-including-how-and-when-assessed}

The outcome of interest is linezolid-induced thrombocytopenia, defined
as (i) a platelet count of \textless{} 112.5 x 10\textsuperscript{9}
cells/L (75\% of the lower limit of normal) for patients with a baseline
platelet count in the normal range; (ii) A reduction in platelet count
of ≥ 25\% from the baseline value for patients with a baseline platelet
count of \textless{} 150 x 10\textsuperscript{9} cells/L {[}4,8,9{]}.

Normal platelet count is defined as 150-450 x 10\textsuperscript{9}
cells/L. Baseline platelet count is defined as the last recorded PLT
value before the start of linezolid therapy. Participants are considered
to have met the outcome if their platelet count value meets the above
criteria at any time during linezolid therapy or within 5 days after the
end of therapy.

\begin{tcolorbox}[enhanced jigsaw, breakable, opacitybacktitle=0.6, leftrule=.75mm, opacityback=0, title=\textcolor{quarto-callout-warning-color}{\faExclamationTriangle}\hspace{0.5em}{Warning}, coltitle=black, bottomtitle=1mm, toprule=.15mm, toptitle=1mm, colback=white, titlerule=0mm, colframe=quarto-callout-warning-color-frame, arc=.35mm, bottomrule=.15mm, rightrule=.15mm, left=2mm, colbacktitle=quarto-callout-warning-color!10!white]

Thrombocytopenia may occur within a few days after stopping LZD, when
the drug hasn't been completely eliminated. However, it is unknown
exactly how long after stopping LZD can a TP event still be attributed
to LZD use. We deemed that any TP events that occur after 5 days of
stopping LZD would not be related to LZD use.

Our rationale is that after 5 days (120 hrs), LZD is guaranteed to be
completely eliminated in all patients, as the longest
t\textsubscript{1/2} is 8.3 ± 2.4 hrs in end-stage renal disease
patients, +3 SD would be \textasciitilde16 hrs, so 120 hrs is
\textgreater7 half-lives, therefore in patients with the worst
clearance, 99\% of them would have 99\% of the drug eliminated from
their systems. Furthermore, trough LZD concentration
(C\textsubscript{min}) has previously been identified as a predictor of
LI-TP development, and LI-TP itself is mostly reversible after
discontinuation, so we would argue that any TP events that occur after
LZD has been eliminated from the system would not be related to LZD use.

\end{tcolorbox}

\subsubsection{6b. Define all predictors used in developing or
validating the model, including how and when
measured}\label{b.-define-all-predictors-used-in-developing-or-validating-the-model-including-how-and-when-measured}

Predictors will be screened for inclusion in the model if they meet all
of the following criteria: (i) has been identified as a risk factor of
LI-TP in previous studies; (ii) can be collected or evaluated from the
information in the datasets; (iii) for concomitant medications, has
drug-induced immune thrombocytopenia as an adverse drug reaction with a
frequency of at least \textgreater{} 1/1000 in the drug label or
Micromedex; (iv) has consensus from a clinical expert panel as possibly
related to LI-TP development.

The following information was extracted from all records:

\begin{itemize}
\tightlist
\item
  Patient demographics
\item
  Clinical department where linezolid was initiated.
\item
  Co-morbidities
\item
  Invasive procedures performed
\item
  Infection type
\item
  Laboratory results
\item
  Linezolid route of administration
\item
  Linezolid dose in milligrams.
\item
  Linezolid duration, defined as the number of days from the first to
  the last dose of linezolid.
\item
  Concomitant medications during linezolid therapy
\end{itemize}

\subsection{Data preparation}\label{data-preparation}

\subsubsection{7a. Describe how the data were prepared for analysis,
including any cleaning, harmonisation, linkage, and quality
checks}\label{a.-describe-how-the-data-were-prepared-for-analysis-including-any-cleaning-harmonisation-linkage-and-quality-checks}

Harmonisation between datasets was mainly done via manually recording
data to a standardized form. Data was then entered into an Excel
spreadsheet. Data cleaning was done by handling duplicates, checking for
missing values and inconsistencies. Multiple linezolid treatment
episodes in the same patient were treated as duplicates and only the
first episode was included in the analysis. Patients with missing values
were excluded from subsequent analyses. Inconsistencies were resolved by
referring back to the original records.

Before analysis, the extracted predictors are limited to those that meet
criteria (i) to (iii) in the previous section:

\begin{itemize}
\tightlist
\item
  Patient demographics were limited to age in years, gender, and weight
  in kilograms.
\item
  Clinical department was recorded into binary variables: intensive care
  unit, emergency department, and others.
\item
  Co-morbidities were recorded into binary variables: hypertension,
  heart failure, angina, myocardial infarction, cerebral vascular
  accident, diabetes, chronic obstructive pulmonary disease, cirrhosis,
  malignancies, and hematological disorders.
\item
  Invasive procedures were recorded into binary variables: endotracheal
  intubation, central venous catheter insertion, intermittent
  hemodialysis, and continuous renal replacement therapy.
\item
  Infection type was recorded into binary variables: community-acquired
  pneumonia, hospital-acquired pneumonia, skin and soft tissue
  infection, central nervous system infection, intra-abdominal
  infection, urinary tract infection, bone and joint infection,
  septicemia, and sepsis.
\item
  Laboratory results were limited to serum creatinine, hemoglobin count,
  white blood cell count, and platelet count. Creatinine clearance was
  estimated from serum creatinine using the Cockcroft-Gault equation.
\item
  Linezolid route of administration was recorded into binary variables:
  intravenous, oral, and both.
\item
  Linezolid dose in milligrams.
\item
  Linezolid duration in days.
\item
  Concomitant medications were recoded to binary variables: carbapenems,
  daptomycin, teicoplanin, levofloxacin, ibuprofen, naproxen, heparin,
  clopidogrel, enoxaparin, eptifibatide, carbamazepine, valproic acid,
  quetiapine, atezolizumab, pembrolizumab, trastuzumab, tacrolimus,
  fluorouracil, irinotecan, leucovorin, oxaliplatin, pyrazinamide, and
  rifampin.
\end{itemize}

\subsubsection{7b. Describe the method for assessing risk of bias and
applicability in the individual clusters (eg, using
PROBAST)}\label{b.-describe-the-method-for-assessing-risk-of-bias-and-applicability-in-the-individual-clusters-eg-using-probast}

\begin{tcolorbox}[enhanced jigsaw, breakable, opacitybacktitle=0.6, leftrule=.75mm, opacityback=0, title=\textcolor{quarto-callout-important-color}{\faExclamation}\hspace{0.5em}{Important}, coltitle=black, bottomtitle=1mm, toprule=.15mm, toptitle=1mm, colback=white, titlerule=0mm, colframe=quarto-callout-important-color-frame, arc=.35mm, bottomrule=.15mm, rightrule=.15mm, left=2mm, colbacktitle=quarto-callout-important-color!10!white]

Is this even possible for this study?

\end{tcolorbox}

\subsubsection{7c. For validation, identify any differences in
definition and measurement from the development data (eg, setting,
eligibility criteria, outcome,
predictors)}\label{c.-for-validation-identify-any-differences-in-definition-and-measurement-from-the-development-data-eg-setting-eligibility-criteria-outcome-predictors}

\subsubsection{7d. Describe how missing data were
handled}\label{d.-describe-how-missing-data-were-handled}

Any subsequent analyses were conducted on the complete case dataset.
There are \textasciitilde5\% of observations with missing values in the
dataset, which is considered low. The missing data mechanism is assumed
to be missing completely at random.

\begin{tcolorbox}[enhanced jigsaw, breakable, opacitybacktitle=0.6, leftrule=.75mm, opacityback=0, title=\textcolor{quarto-callout-important-color}{\faExclamation}\hspace{0.5em}{Important}, coltitle=black, bottomtitle=1mm, toprule=.15mm, toptitle=1mm, colback=white, titlerule=0mm, colframe=quarto-callout-important-color-frame, arc=.35mm, bottomrule=.15mm, rightrule=.15mm, left=2mm, colbacktitle=quarto-callout-important-color!10!white]

What is a possible reason for missing data?

\end{tcolorbox}

\subsection{Data analysis}\label{data-analysis}

\subsubsection{8a. Describe how predictors were handled in the
analyses}\label{a.-describe-how-predictors-were-handled-in-the-analyses}

\subsubsection{8b. Specify the type of model, all model building
procedures (eg, any predictor selection and penalisation), and method
for
validation}\label{b.-specify-the-type-of-model-all-model-building-procedures-eg-any-predictor-selection-and-penalisation-and-method-for-validation}

\subsubsection{8c. Describe how any heterogeneity across clusters (eg,
studies or settings) in model parameter values was
handled}\label{c.-describe-how-any-heterogeneity-across-clusters-eg-studies-or-settings-in-model-parameter-values-was-handled}

\subsubsection{8d. For validation, describe how the predictions were
calculated}\label{d.-for-validation-describe-how-the-predictions-were-calculated}

\subsubsection{8e. Specify all measures used to assess model performance
(eg, calibration, discrimination, and decision curve analysis) and, if
relevant, to compare multiple
models}\label{e.-specify-all-measures-used-to-assess-model-performance-eg-calibration-discrimination-and-decision-curve-analysis-and-if-relevant-to-compare-multiple-models}

\subsubsection{8f. Describe how any heterogeneity across clusters (eg,
studies or settings) in model performance was handled and
quantified}\label{f.-describe-how-any-heterogeneity-across-clusters-eg-studies-or-settings-in-model-performance-was-handled-and-quantified}

\subsubsection{8g. Describe any model updating (eg, recalibration)
arising from the validation, either overall or for particular
populations or
settings}\label{g.-describe-any-model-updating-eg-recalibration-arising-from-the-validation-either-overall-or-for-particular-populations-or-settings}

\subsection{Sensitivity analysis}\label{sensitivity-analysis}

\section{Results}\label{results}

\subsection{Participants and datasets}\label{participants-and-datasets}

\subsection{Risk of bias}\label{risk-of-bias}

\subsection{Model development and
specification}\label{model-development-and-specification}

\subsection{Model performance}\label{model-performance}

\subsection{Model updating}\label{model-updating}

\subsection{Sensitivity analysis}\label{sensitivity-analysis-1}

\section{Discussion}\label{discussion}

\subsection{Interpretation}\label{interpretation}

\subsection{Implications}\label{implications}

\section{Other information}\label{other-information}

\subsection{Supplementary information}\label{supplementary-information}

\subsection{Funding}\label{funding}

\subsection{References}\label{references}

\phantomsection\label{refs}
\begin{CSLReferences}{0}{1}
\bibitem[\citeproctext]{ref-liu_analysis_2021}
1. Liu Y, Liu T, Wei G, Yan P, Fang X, Xie L. Analysis of risk factors
and establishment of risk prediction model for linezolid-associated
thrombocytopenia. Medical Journal of Chinese People's Liberation Army
{[}Internet{]}. 2021;46. Available from:
\url{https://d.wanfangdata.com.cn/periodical/jfjyxzz202108006}

\bibitem[\citeproctext]{ref-duan_regression_2022}
2. Duan L, Zhou Q, Feng Z, Zhu C, Cai Y, Wang S, et al. A {Regression}
{Model} to {Predict} {Linezolid} {Induced} {Thrombocytopenia} in
{Neonatal} {Sepsis} {Patients}: {A} {Ten}-{Year} {Retrospective}
{Cohort} {Study}. Front Pharmacol {[}Internet{]}. 2022;13:710099.
Available from: \url{https://www.ncbi.nlm.nih.gov/pubmed/35185555}

\bibitem[\citeproctext]{ref-qin_development_2021}
3. Qin Y, Chen Z, Gao S, Pan MK, Li YX, Lv ZQ, et al. Development and
{Validation} of a {Risk} {Prediction} {Model} of {Linezolid}-induced
{Thrombocytopenia} in {Elderly} {Patients} {[}Internet{]}. In Review;
2021 Jun. Available from:
\url{https://www.researchsquare.com/article/rs-582799/v1}

\bibitem[\citeproctext]{ref-xu_establishment_2023}
4. Xu J, Lu J, Yuan Y, Duan L, Shi L, Chen F, et al.
\href{https://doi.org/10.1093/jac/dkad191}{Establishment and validation
of a risk prediction model incorporating concentrations of linezolid and
its metabolite {PNU142300} for linezolid-induced thrombocytopenia}. The
Journal of Antimicrobial Chemotherapy. 2023;78:1974--81.

\bibitem[\citeproctext]{ref-peduzzi1995}
5. Peduzzi P, Concato J, Feinstein AR, Holford TR. Importance of events
per independent variable in proportional hazards regression analysis II.
Accuracy and precision of regression estimates. Journal of Clinical
Epidemiology {[}Internet{]}. 1995;48:1503--10. Available from:
\url{https://www.jclinepi.com/article/0895-4356(95)00048-8/abstract}

\bibitem[\citeproctext]{ref-riley2019}
6. Riley RD, Snell KI, Ensor J, Burke DL, Harrell Jr FE, Moons KG, et
al. Minimum sample size for developing a multivariable prediction model:
PART II - binary and time-to-event outcomes. Statistics in Medicine
{[}Internet{]}. 2019;38:1276--96. Available from:
\url{https://onlinelibrary.wiley.com/doi/abs/10.1002/sim.7992}

\bibitem[\citeproctext]{ref-zhao_prediction_2024}
7. Zhao X, Peng Q, Hu D, Li W, Ji Q, Dong Q, et al. Prediction of risk
factors for linezolid-induced thrombocytopenia based on neural network
model. Frontiers in Pharmacology {[}Internet{]}. 2024 {[}cited 2024 Feb
27{]};15. Available from:
\url{https://www.frontiersin.org/journals/pharmacology/articles/10.3389/fphar.2024.1292828}

\bibitem[\citeproctext]{ref-zyvoxpr}
8. Zyvox prescribing information {[}Internet{]}. Available from:
\url{https://labeling.pfizer.com/showlabeling.aspx?id=649}

\bibitem[\citeproctext]{ref-kawasuji_proposal_2021}
9. Kawasuji H, Tsuji Y, Ogami C, Kimoto K, Ueno A, Miyajima Y, et al.
Proposal of initial and maintenance dosing regimens with linezolid for
renal impairment patients. BMC Pharmacology and Toxicology
{[}Internet{]}. 2021 {[}cited 2024 Feb 26{]};22:13. Available from:
\url{https://doi.org/10.1186/s40360-021-00479-w}

\end{CSLReferences}



\end{document}
